% !TeX root = RJwrapper.tex
\title{Introducing parmsurvfit Package - Simple Parametric Survival Analysis
with R}
\author{by Victor Wilson, Ashley Jacobson, Shannon Pileggi}

\maketitle

\abstract{%
This article introduces the R package \CRANpkg{parmsurvfit}, which
executes basic parametric survival analysis techniques similar to those
in Minitab. Among these are plotting hazard, cumulative hazard,
survival, and density curves, computing survival probabilites, and
computing summary statistics based on a specified parametric
distribution. We describe appropriate usage of these functions,
interpretation of output, and provide examples of how to utilize these
functions in real-world datasets.
}

% Any extra LaTeX you need in the preamble

\hypertarget{introduction}{%
\section{Introduction}\label{introduction}}

Survival analysis is a branch of statistics that primarily deals with
analyzing the time until an event of interest occurs. This event could
be a variety of different endpoints such as death, development of
disease, or first score of a soccer game. Observations in survival
analysis may also be described as censored, which occurs when an
observation's actual time to event is unknown, but is known to be within
some specified range. For example, if a subject does not experience the
event of interest prior to study completion, then their ``observed''
event time (time until study completion) is less than their ``actual''
event time. This particular case is known as right-censoring, which is
the most common form. Right censoring can also occur if a subject drops
out before the end of the study and does not experience the event of
interest.

Typically, survival analysis begins with computing what are known as the
survival and hazard functions \citep{Kleinbaum2012}. The survival
function estimates the probability of surviving (also known as not
experiencing the event of interest) beyond any giventime \(t\). The
hazard function computes the conditional probability that the subject
will experience the event in the next instance of time, given that they
have survived up until the specified point in time. Summary statistics
that are commonly reported include median survival time, mean survival
time, and percentiles of survival time. These functions and statistics
can be computed by either parametric or nonparametric techniques. In the
\pkg{parmsurvfit} package, all of the functions that we developed
utilize parametric methods of survival analysis, which assumes that the
distribution of the survival times follows a known probability
distribution.

R has many survival packages that address non-parametric survival
analysis, such as the \CRANpkg{survival} package. However, prior to the
\pkg{parmsurvfit} package, parametric estimation of survival functions
and statistics was not easily attained. The \pkg{parmsurvfit} package
attempts to emulate the ease and functionality of parametric survival
analysis features available in Minitab.

This paper describes example survival data available in the
\pkg{parmsurvfit} package, the functions that the package contains, how
the data is formatted in order to utilize these functions, and what the
output of these functions represent. There are three major groups of
functions that we created: assessing fit, survival functions (density,
survival, hazard, cumulative hazard), and computing statistics (mean,
median, survival probabilities). Explanations are presented according to
these groups of functions.

\hypertarget{data}{%
\section{Data}\label{data}}

The \pkg{parmsurvfit} package contains five data sets with observations
subject to right-censoring (\code{aggressive}, \code{firstdrink},
\code{graduate}, \code{oscars}, and \code{rearrest}). Subsequent
examples in this paper are based on the \code{firstdrink} data set,
which contains 1,000 records from the 1990-1992 National Comorbidity
Survey regarding age at first drink of alcohol (\code{age}). An
observation is recorded as complete (\code{censor = 1}) if the age at
first drink of alcohol is known. An observation is recorded as
incomplete, or right-censored (\code{censor = 0}) if the subject had not
yet consumed an alcoholic beverage at the time of the study execution.
In this case, the subject's ``actual'' event time (age at first drink)
is only known to exceed their ``observed'' event time (age at time of
study). This data set also includes a \code{gender} variable such that
\code{1} corresponds to males and \code{2} corresponds to females.

\hypertarget{fitting-right-censored-survival-data}{%
\section{Fitting right-censored survival
data}\label{fitting-right-censored-survival-data}}

The \code{fit\_data} function produces maximum likelihood estimates
(MLE) for right censored data based on the input distribution. The
\code{fit\_data} function utilizes the \code{fitdistcens} function in
the \CRANpkg{fitdistrplus} package, but allows for more intuitive input
of right-censored data than as specified with \code{fitdistcens} (which
allows input of other types of censoring). The \code{fit\_data} function
in the \pkg{parmsurvfit} package inputs two variables: \code{time} and
\code{censor}. The \code{time} variable contains the time-to-event
variable, while the \code{censor} variable indicates whether right
censoring is present (0 corresponds to censored data and 1 corresponds
to complete data). Furthermore, the user specifies the desired
parametric distribution in \code{dist} by inputting the base name of the
distribution as a character string. For example, to utilize the normal
distribution you would specify \samp{norm} as it is the base of
\code{dnorm}, \code{pnorm}, etc. Commonly utilized distributions for
survival analysis include Weibull (\samp{weibull}), log-normal
(\samp{lnorm}), exponential (\samp{exp}), and logistic (\samp{logis})
distributions. The function also takes in an optional grouping variable,
which fits the data for each group individually. The function returns an
object of class \samp{fitdistcens}, and if there is a grouping variable
it returns a list of objects of class \samp{fitdistcens}.

\hypertarget{example}{%
\subsection{Example}\label{example}}

In this example, we fit the Weibull distribution to the
\file{firstdrink} data set where the time to event variable is
\code{age} and the variable that indicates censoring status is
\code{censor}. The maximum likelihood estimates of the location and
scale parameters are returned.

\begin{Schunk}
\begin{Sinput}
library(parmsurvfit)
fit_data(data = firstdrink, dist = "weibull", time = "age", censor = "censor")
\end{Sinput}
\begin{Soutput}
#> Fitting of the distribution ' weibull ' on censored data by maximum likelihood 
#> Parameters:
#>        estimate
#> shape  2.536106
#> scale 19.684061
#> Fixed parameters:
#> data frame with 0 columns and 0 rows
\end{Soutput}
\end{Schunk}

\hypertarget{assessing-fit}{%
\section{Assessing fit}\label{assessing-fit}}

Since all of the functions available in this package assume that the
survival data follows a known parametric distribution, we presents
methods to evaluate how well the assumed model fits the data. Some
common methods used to assess goodness of fit are viewing a histogram of
the data, Q-Q (Quantile-Quantile) plots, the Anderson-Darling test
statistic.

\hypertarget{histograms-with-density-curves}{%
\subsection{Histograms with density
curves}\label{histograms-with-density-curves}}

The \code{plot\_density} function produces histograms with overlayed
density curves to allow a user to visually assess fit of a parametric
distribution to data. Parameter estimates for the specified parametric
distribution are provided as well. This function also supports the
ability to plot separate histograms and density functions for each level
of a grouping variable. Below, we fit the weibull distribution to age
until first drink by gender. Three plots are produced, each based on
their respective MLE's: a plot for males (\code{level = 1}), females
(\code{level = 2}), and overall.\\
In these plots, all time to event data are plotted regardless of
censoring status.

\begin{Schunk}
\begin{Sinput}
plot_density(data = firstdrink, dist = "weibull", time = "age", censor = "censor", by = "gender")
\end{Sinput}

\includegraphics{Simple_Survival_Analysis2_files/figure-latex/densityplot-1} 
\includegraphics{Simple_Survival_Analysis2_files/figure-latex/densityplot-2} 
\includegraphics{Simple_Survival_Analysis2_files/figure-latex/densityplot-3} \end{Schunk}

Here, we observe that the Weibull distribution could possibly be too
flat for the distribution of age until first drink of alcohol.

\hypertarget{q-q-plots}{%
\subsection{Q-Q plots}\label{q-q-plots}}

\textbf{Quantiles vs percentileS!!!}

The \code{plot\_qqsurv} function creates a quantile-quantile plot of
right-censored data given that it follows a specified distribution. In
typical Q-Q plots the hypothesized (theoretical) quantiles are plotted
on the \(y\)-axis and the observed (sample) quantiles are plotted on the
\(x\)-axis. A \(y=x\) line is included in these plots, because if the
observed data fit the hypothesized distribution perfectly, all of the
points would lie exactly on this diagonal line. Here, the points are
plotted according to the median rank method \citep{Minitabqq} to
accommodate the right-censoring features.

\begin{Schunk}
\begin{Sinput}
plot_qqsurv(data = firstdrink, dist = "weibull", time = "age", censor = "censor")
\end{Sinput}

\includegraphics{Simple_Survival_Analysis2_files/figure-latex/qqplot-1} \end{Schunk}

As seen in the Q-Q plot, there are some deviations from the provided
\(y=x\) line, indicating that a Weibull distribution may not be an ideal
fit for the data.

\hypertarget{anderson-darling-test-statistic}{%
\subsection{Anderson-Darling test
statistic}\label{anderson-darling-test-statistic}}

While creating Q-Q plots are a great way to visualize how a particular
distribution may fit the data, it can be difficult at times to
definitively decide how well the plot fits the data. The
\code{compute\_AD} function computes the Anderson-Darling (AD) test
statistic, which provides a numerical test statistic that measures how
well a particular distribution the data fits such that lower values
indicate a better fit. Computation of the test statistic adhered to
Minitab's documentation, utilizing the median rank plotting method
\citep{Minitabgof}.

\begin{Schunk}
\begin{Sinput}
compute_AD(data = firstdrink, dist = "weibull", time = "age", censor = "censor")
\end{Sinput}
\begin{Soutput}
#> [1] 315.5693
\end{Soutput}
\end{Schunk}

Here, the AD test statistic value observed of 315.5693 provides no
useful information as a stand-alone value, but rather can be used to
judge fit relatively to another computed AD test statistic.

\hypertarget{survival-functions}{%
\section{Survival functions}\label{survival-functions}}

This section provides an overview of the survival functions available in
this package. Some of the most common functions used in survival
analysis are the survival function, the hazard function, and the
cumulative hazard function. Table \ref{table:functions} lists each
function and its relationship to the other functions, as well as the
formula used to compute each function.

\begin{table}
\begin{tabular}{ll}
\hline
Function & Relationships  \\
\hline
PDF & ${f(t)=\frac{d}{dt}F(t)}$\\
CDF  & ${F(t)=\int_0^t f(y)dy}$\\
Survival & ${S(t)=1-F(t)=\exp[-H(t)]=\exp[-\int_0^th(y)dy]}$ \\
Hazard & ${h(t)=f(t)/S(t)=-\frac{d}{dt}\ln[S(t)]}$ \\
Cumulative hazard & ${H(t)=\int_0^t h(y)dy=-\ln[S(t)]}$\\
\hline
\end{tabular}
\label{table:functions}
\end{table}

\hypertarget{survival-plots}{%
\subsection{Survival plots}\label{survival-plots}}

The survival function \(S(t)\) estimates the proportion of subjects that
survive beyond a specified time \(t\). The \code{plot\_surv} function
plots the survival curve of right censored data given a specified
parametric distribution. This function also provides the option to plot
by a grouping variable, which if specified, displays separate curves for
each value of the specified grouping variable. In these plots, survival
time is plotted on the \(x\)-axis, while survival probability is plotted
on the \(y\)-axis.

\begin{Schunk}
\begin{Sinput}
plot_surv(data = firstdrink, dist = "weibull", time = "age", censor = "censor", by = "gender")
\end{Sinput}

\includegraphics{Simple_Survival_Analysis2_files/figure-latex/surv-1} \end{Schunk}

In this example, we fit a Weibull distribution to the \code{firstdrink}
dataset, grouping by the \code{gender} variable. Here, the blue line
represents the estimated survival curve for males (\code{group = 1}),
while the red line represents the estimated survival curve for females
(\code{group = 2}). From this plot, we see that the survival curve for
females is consistently above the survival curve for males throughout
all points in time. Due to this, we can conclude that males tend to
experience their first drink of alcohol before females do.

\hypertarget{hazard-function}{%
\subsection{Hazard function}\label{hazard-function}}

The hazard function, denoted \(h(t)\), esimates the conditional risk
that a subject will experience the event of interest in the next instant
of time, given that the subject has survived beyond a certain time
\(t\). For example, if we observe that a person has survived for 17
years without first trying alcohol, the hazard function would estimate
the risk that the person will experience their first drink of alcohol in
the next short instant of time, based on the fact that the person has
already survived 17 years alcohol free. However, hazard is a rate and
not probability, and therefore can take values greater than one.
Moreover, the hazard function can be both increasing or decreasing. The
\code{plot\_haz} function plots the hazard function based on specified
parametric distribution, with the option to include a grouping variable.

\begin{Schunk}
\begin{Sinput}
plot_haz(data = firstdrink, dist = "weibull", time = "age", censor = "censor",  by = "gender")
\end{Sinput}

\includegraphics{Simple_Survival_Analysis2_files/figure-latex/haz-1} \end{Schunk}

Based on the estimated hazard function, we see that as males and females
continue to ``survive'' (not yet experience their first drink of
alcohol), their risk of experiencing the event of interest (drinking) in
the next instant of time increases.\\
This risk increases more dramatically for males relative to females.

\hypertarget{cumulative-hazard-function}{%
\subsection{Cumulative hazard
function}\label{cumulative-hazard-function}}

While hazard plots are usually useful in assessing a subject's risk of
experiencing the event of interest in the next moment of time, these
plots can be difficult to read and understand at times. Sometimes, the
changes in hazard are very subtle, making it difficult to describe
periods of increasing and decreasing risk. In order to accurately assess
how hazard rates change over time, we investigate the accumulation of
hazard rates over time, known as cumulative hazard. The cumulative
hazard function, denoted \(H(t)\), is the total accumulated risk of
experiencing an event up to time \(t\).

Since the cumulative hazard function is an accumulation of rates, it is
important to note that this function is non-decreasing and is hardly
ever remains constant by nature. The \code{plot\_cumhaz} function
displays cumulative hazard plots, given that the data follows a
specified parametric distribution.

\begin{Schunk}
\begin{Sinput}
plot_cumhaz(data = firstdrink, dist = "weibull", time = "age", censor = "censor",  by = "gender")
\end{Sinput}

\includegraphics{Simple_Survival_Analysis2_files/figure-latex/cumhaz-1} \end{Schunk}

As expected, the cumulative hazard function is increasing for both males
and females. Here, the total accumulated risk of experiencing the first
drink of alcohol is consistently greater for males compared to females.

\hypertarget{probabilities-and-statistics}{%
\section{Probabilities and
Statistics}\label{probabilities-and-statistics}}

Survival analysis is typically accompanied by reporting summary
statistics such as the mean, median, standard deviation, and percentiles
of survival time. In addition, it is also common to compute survival
probabilities.

\hypertarget{survival-probability}{%
\subsection{Survival probability}\label{survival-probability}}

A survival probability estimates the probability that a subject survives
(does not experience the event of interest) beyond a specified time
\(t\). The function \code{surv\_prob} computes the probability of
survival beyond time \(t\), given that the data follows a specified
parametric distribution. The \code{num} argument specifies the time
\(t\) of interest. The default for \code{surv\_prob} is to compute an
upper tail probability; this can be reversed to a lower tail probability
useing the \code{lower.tail} argument. The function can also optionally
take a grouping variable with the \code{by} argument.

\begin{Schunk}
\begin{Sinput}
surv_prob(data = firstdrink, dist = "weibull", num = 30, lower.tail = F, time = "age", censor = "censor", by = "gender")
\end{Sinput}
\begin{Soutput}
#> 
#> For level = 1 
#> P(T > 30) = 0.02488195
#> 
#> For level = 2 
#> P(T > 30) = 0.08227309
#> 
#> For all levels
#> P(T > 30) = 0.05439142
\end{Soutput}
\end{Schunk}

Given a Weibull distribution, the overall estimated probability that a
person survives beyond 30 years without having their first drink of
alcohol is approximately 0.05. However, males (\code{level = 1}) are
less likely to survive beyond 30 years compared to females
(\code{level = 2}).

\hypertarget{summary-statistics}{%
\subsection{Summary statistics}\label{summary-statistics}}

The \code{surv\_summary} function estimates various summary statistics,
including mean, median, standard deviation, and percentiles of survival
time given that the data follows a specified parametric distribution.
This function also supports the option to provide separate summary
statistics for each level of a grouping variable, if desired All summary
statistics from the class \samp{fitdistcens} are provided. If the
distribution supplied is one of normal, lognormal, exponential, weibull,
or logistic then the standard deviation reported is an exact computation
from parameter estimates However, if a user specifies a distribution
other than that from this list, then the standard deviation is estimated
from 1,000 randomly generated values from the distribution.

\begin{Schunk}
\begin{Sinput}
surv_summary(data = firstdrink, dist = "weibull", time = "age", censor = "censor", by = "gender")
\end{Sinput}
\begin{Soutput}
#> 
#> 
#> For level = 1 
#> shape        2.637645
#> scale        18.2804
#> Log Liklihood    -1425.271
#> AIC      2854.541
#> BIC      2862.808
#> Mean     16.24398
#> StDev        6.625303
#> First Quantile   11.39844
#> Median       15.90884
#> Third Quantile   20.6903
#> 
#> For level = 2 
#> shape        2.516025
#> scale        20.85053
#> Log Liklihood    -1730.273
#> AIC      3464.546
#> BIC      3473.126
#> Mean     18.50288
#> StDev        7.872356
#> First Quantile   12.70752
#> Median       18.02407
#> Third Quantile   23.74094
\end{Soutput}
\end{Schunk}

Utilizing the grouping variable of gender, produces two separate tables
of summary statistics for males and females, respectively. The mean
survival time for males (16.2 years) is less than the mean survival time
for females (18.5 years).

\hypertarget{conclusion}{%
\section{Conclusion}\label{conclusion}}

The \pkg{parmsurvfit} package allows for parametric survival analysis
methods involving right-censored data to be easily computed in R. The
overall goal of developing this package was to provide a central package
for R users to implement typical methods used in parametric survival
analysis such as computing survival probabilities, creating survival and
hazard plots, and assessing goodness of fit of a parametric distribution
to a dataset.

\hypertarget{acknowledgments}{%
\section{Acknowledgments}\label{acknowledgments}}

This package development was funded by the Bill and Linda Frost fund at
California Polytechnic State University, San Luis Obispo. We would also
like to thank Jeff Sklar for his input and guidance.

\bibliography{RJreferences}


\address{%
Victor Wilson\\
California Polytechnic State University, San Luis Obispo - Statistics
Department\\
\\
}
\href{mailto:victorjw26@yahoo.com}{\nolinkurl{victorjw26@yahoo.com}}

\address{%
Ashley Jacobson\\
California Polytechnic State University, San Luis Obispo - Statistics
Department\\
\\
}
\href{mailto:ashleypjacobson@gmail.com}{\nolinkurl{ashleypjacobson@gmail.com}}

\address{%
Shannon Pileggi\\
California Polytechnic State University, San Luis Obispo - Statistics
Department\\
\\
}
\href{mailto:spileggi@calpoly.edu}{\nolinkurl{spileggi@calpoly.edu}}

