% !TeX root = RJwrapper.tex
\title{Introducting survfuncs Package - Simple Survival Analysis with R}
\author{by Victor Wilson, Ashley Jacobson}

\maketitle

\abstract{%
This article introduces the R package survfuncs, which executes
parametric survival analysis techniques similar to those in Minitab. The
functions available in this package carry out basic survival analysis
techniques. Among these are plotting hazard, cumulative hazard, and
survival curves, based on specified parametric distributions, computing
survival probabilites, and computing summary statistics. We describe
appropriate usage of these functions, what the output means, and provide
examples of how to utilizie this functions in real-world datasets.
}

% Any extra LaTeX you need in the preamble

\hypertarget{introduction}{%
\subsection{Introduction}\label{introduction}}

The major goal in survival analysis is analyzing the time until an event
of interest occurs. This event could be a variety of different things
such as death, development of disease, or first score of a soccer game.
Observations in survival analysis may also be desrcibed as censored,
which occurs when an observation's survival time is incomplete. The most
common way that this occurs is through right censoring, which occurs
when a subject does not experience the event of interest within the
duration of the study. Right censoring can also occur if a subject drops
out before the end of the study and does not experience the event of
interest. Due to the inherent issue of censoring that is typically found
in datasets involving survival analysis, computations and analyses can
be difficult to carry out with many standard functions avaiable in R, as
the majority of these do not account for censored data. The censored
data here is of value and we cannot merely eliminate the observations
which have censored data.

Some of the most popular techniques and statistics utilized when
carrying out a survival analysis are computing what are known as the
survival and hazard functions. The survival function is important
because it gives the proability of surviving (also known as not
experiencing the event of interest), for any given time. Similarly, the
hazard function is also useful to compute because it gives the
conditional probability that the subject will experience the event in
the next instance of time, given that they have survived up until the
specified point in time. Other popular statistics that are utilized are
median survial time, mean survival time, and percentiles of survival
time. In this package, all of the functions that we developed utilize
parametric methods of survival anlaysis, which assumes that the
distribution of the survival times follows a known probability
distribution.

Currently, R does have many survival packages that have been developed
that compute some of these statistics. However, we noticed that Minitab
has very concise and easy to utilize functions for computing and
displaying many of these survival statistics and plots, but this same
output is not readily available in any single one package in R, or in
some cases not available in any at all. Thus, we decided to develop a
package that emulates the output found in Minitab for survival analysis,
which contains all of these commonly utilized statistics and plots.

This paper describes the functions that this package contains, how the
data is formatted in order to utilize these functions, and what the
output of these functions represent. There are 3 major groups of
functions that we created: fitting the censored data, displaying plots
(hazard, cumulative hazard, and survival), and computing summary
statistics (mean, median, survival probabilities). The majority of this
paper will be organized following these groups of functions.

\hypertarget{section-title-in-sentence-case}{%
\subsection{Section title in sentence
case}\label{section-title-in-sentence-case}}

This section may contain a figure such as Figure \ref{figure:rlogo}.

\begin{figure}[htbp]
  \centering
  \includegraphics{Rlogo}
  \caption{The logo of R.}
  \label{figure:rlogo}
\end{figure}

\hypertarget{another-section}{%
\subsection{Another section}\label{another-section}}

There will likely be several sections, perhaps including code snippets,
such as:

\begin{Schunk}
\begin{Sinput}
x <- 1:10
x
\end{Sinput}
\begin{Soutput}
#>  [1]  1  2  3  4  5  6  7  8  9 10
\end{Soutput}
\end{Schunk}

\hypertarget{summary}{%
\subsection{Summary}\label{summary}}

This file is only a basic article template. For full details of
\emph{The R Journal} style and information on how to prepare your
article for submission, see the
\href{https://journal.r-project.org/share/author-guide.pdf}{Instructions
for Authors}.

\bibliography{RJreferences}


\address{%
Victor Wilson\\
California Polytechnic State University, San Luis Obispo - Statistics
Department\\
\\
}
\href{mailto:victorjw26@yahoo.com}{\nolinkurl{victorjw26@yahoo.com}}

\address{%
Ashley Jacobson\\
California Polytechnic State University, San Luis Obispo - Statistics
Department\\
\\
}
\href{mailto:ashleypjacobson@gmail.com}{\nolinkurl{ashleypjacobson@gmail.com}}

