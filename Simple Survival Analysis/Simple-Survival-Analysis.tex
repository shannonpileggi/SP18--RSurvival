% !TeX root = RJwrapper.tex
\title{Introducing parmsurvfit Package - Simple Parametric Survival Analysis
with R}
\author{by Victor Wilson, Ashley Jacobson}

\maketitle

\abstract{%
This article introduces the R package parmsurvfit, which executes
parametric survival analysis techniques similar to those in Minitab. The
functions available in this package carry out basic survival analysis
techniques. Among these are plotting hazard, cumulative hazard, and
survival curves, based on specified parametric distributions, computing
survival probabilites, and computing summary statistics. We describe
appropriate usage of these functions, what the output means, and provide
examples of how to utilizie this functions in real-world datasets.
}

% Any extra LaTeX you need in the preamble

\hypertarget{introduction}{%
\subsection{Introduction}\label{introduction}}

Survival analysis is a branch of statistics that primarily deals with
analyzing the time until an event of interest occurs. This event could
be a variety of different things such as death, development of disease,
or first score of a soccer game. Observations in survival analysis may
also be desrcibed as censored, which occurs when an observation's
survival time is incomplete. The most common way that this occurs is
through right censoring, which occurs when a subject does not experience
the event of interest within the duration of the study. Right censoring
can also occur if a subject drops out before the end of the study and
does not experience the event of interest. Due to the inherent issue of
censoring that is typically found in datasets involving survival
analysis, computations and analyses can be difficult to carry out with
many standard functions avaiable in R, as the majority of these do not
account for censored data. The censored data here is of value and we
cannot merely eliminate the observations which have censored data.

Some of the most popular techniques and statistics utilized when
carrying out a survival analysis are computing what are known as the
survival and hazard functions. The survival function is important
because it gives the proability of surviving (also known as not
experiencing the event of interest), for any given time. Similarly, the
hazard function is also useful to compute because it gives the
conditional probability that the subject will experience the event in
the next instance of time, given that they have survived up until the
specified point in time. Other popular statistics that are utilized are
median survial time, mean survival time, and percentiles of survival
time. In this package, all of the functions that we developed utilize
parametric methods of survival anlaysis, which assumes that the
distribution of the survival times follows a known probability
distribution.

Currently, R does have many survival packages that have been developed
that compute some of these statistics. However, we noticed that Minitab
has very concise and easy to utilize functions for computing and
displaying many of these survival statistics and plots, but this same
output is not readily available in any single one package in R, or in
some cases not available at all. Thus, we decided to develop a package
that emulates the output found in Minitab for survival analysis, which
contains all of these commonly utilized statistics and plots.

This paper describes the functions that this package contains, how the
data is formatted in order to utilize these functions, and what the
output of these functions represent. There are 3 major groups of
functions that we created: fitting the censored data, displaying plots
(hazard, cumulative hazard, and survival), and computing statistics
(mean, median, survival probabilities). The majority of this paper will
be organized following these groups of functions.

\hypertarget{fitting-right-censored-survival-data}{%
\subsection{Fitting Right Censored Survival
Data}\label{fitting-right-censored-survival-data}}

As mentoined previously, this function is very similar to the function
fitdistcens found in the
\href{https://cran.r-project.org/web/packages/fitdistrplus/index.html}{fitdistrplus}
package. This package works to help fit a parametric distribution to
data that is right censored. This function also computes the Maximum
Likelihood Estimate (MLE).

\hypertarget{example}{%
\subsubsection{Example}\label{example}}

\begin{Schunk}
\begin{Soutput}
#> Fitting of the distribution ' logis ' on censored data by maximum likelihood 
#> Parameters:
#>           estimate
#> location 133.74054
#> scale     20.48286
#> Fixed parameters:
#> data frame with 0 columns and 0 rows
\end{Soutput}
\end{Schunk}

\hypertarget{displaying-plots}{%
\subsection{Displaying Plots}\label{displaying-plots}}

This section introduces an overview of the many types of plots that are
available to be displayed via this package. Some of the most common
plots used in Survival Analysis are survival plots, hazard plots,
probability density plots, and cumulative hazard plots. We designed
these functions with an intent to have the output displayed be very
similar to the output that Minitab displays for these types of plots.

\hypertarget{survival-plots}{%
\subsubsection{Survival Plots}\label{survival-plots}}

Survival plots are used to estimate the proportion of subjects that
survive beyond a specified time \textbf{t}. We were motivated to create
the function plot\_surv function in an attempt to mimic the hazard plots
that are available in Minitab. This function plots the survival curve of
right censored data given that it follows a specified parametric
distribution. Some examples of the distributions that this function
supports are the Weibull, Log-Normal, Exponential, Normal, and Logistic
distributions. This function also provides the option to plot by a
grouping variable, which if specified, displays seperate survival plots
for each group of the specified variable.

\textbf{INSERT SURVIVAL FUNCTION EQUATION}

\begin{Schunk}

\includegraphics{Simple_Survival_Analysis_files/figure-latex/unnamed-chunk-2-1} \end{Schunk}

In this example, we fit a Weibull distribution to the rats dataset
available in the \CRANpkg{survival} package, grouping by the ``sex''
variable. The rats dataset contains 300 observations, with 3 rats each
being selected from 100 litters and 1 rat in each litter being
adminstered a drug. The event of interest in this study was whether or
not a rat developed a tumor following the beginning of the study. For
each rat, the litter number (1-100), type of treatment received (coded
as 1 = drug, 0 = control), time until development of tumor or last
follow-up (measured in days), final event status (1 = tumor, 0 =
censored), and sex were recorded. Below is an excerpt of the data.frame.

\begin{Schunk}
\begin{Sinput}
library(survival)
data(rats)
head(rats)
\end{Sinput}
\begin{Soutput}
#>   litter rx time status sex
#> 1      1  1  101      0   f
#> 2      1  0   49      1   f
#> 3      1  0  104      0   f
#> 4      2  1   91      0   m
#> 5      2  0  104      0   m
#> 6      2  0  102      0   m
\end{Soutput}
\end{Schunk}

For example, the rat represented in line 2 came from Litter 1, did not
receive the drug, experienced the development of a tumor at 49 days, and
was a female rat.

As seen in the plot above, two different survival curves were plotted.
The blue line represents the estimated survival curve for male rats,
while the red line represents the estimated survival curve for female
rats. From this plot, we see that the survival curve for male rats is
consistently above the survival curve for female rats throughout all
points in time. Due to this, we can conclude that female rats tend to
experience the event of interest much more quickly than male rats. This
can also be interpreted as male rats tend to survive longer than female
rats in this study.

\hypertarget{hazard-plots}{%
\subsubsection{Hazard plots}\label{hazard-plots}}

Hazard plots, on the other hand, are used to display the conditional
risk that a subject will experience the event of interest in the next
instant of time, given that the subject has survived beyond a certain
amount of time. Essentially, the hazard function attempts to assess the
risk that an individual who has not yet experienced the event in the
very next small amount of time. For example, if we observe that a rat
has survived for 75 days already, the hazard function would estimate the
risk that the rat will die in the next short instant of time, based on
the fact that it has already survived 75 days. We created the
``plot\_haz'' function in order to easily plot hazard functions given
that it follows a specified parametric distribution, with the option to
include a grouping variable.

\textbf{INSERT HAZARD FUNCTION FORMULA}

\begin{Schunk}

\includegraphics{Simple_Survival_Analysis_files/figure-latex/unnamed-chunk-4-1} \end{Schunk}

From this plot above, also using the rats dataset available in the
survival package, we can see that as female rats continue to survive,
their risk of experiencing the event of interest in the next instant of
time dramatically increases. Contrastly, male rats do not seem to have a
greater risk of experiencing the event of interest as they survive
longer. This is demonstrated by the blue line being mostly flat across
all points of time.

\hypertarget{cumulative-hazard-plots}{%
\subsubsection{Cumulative Hazard Plots}\label{cumulative-hazard-plots}}

While hazard plots are usually useful in assessing a subject's risk of
experiencing the event of interest in the next moment of time, these
plots can be difficult to read and understand at times. Sometimes, the
changes in hazard are very subtle, making it difficult to describe
periods of increasing and decreasing risk. In order to accurately assess
how hazard rates change over time, we investigate the accumulation of
hazard rates over time, known as cumulative hazard. The cumulative
hazard function, denoted \textbf{H(t)}, is the accumulated risk of
experiencing an event up to time \textbf{t}. Since the cumulative hazard
function is an accumulation of rates, it is important to note that this
function is non-decreasing and is hardly ever remains constant by
nature. We developed the function ``plot\_cumhaz'' in order to easily
display cumulative hazard plots, given that the data follows a specified
parametric distribution. The functionality of this function is nearly
identical to that of ``plot\_haz'', with the only distinction being that
it plots cumulative hazard curves instead of hazard curves.

\begin{Schunk}

\includegraphics{Simple_Survival_Analysis_files/figure-latex/unnamed-chunk-5-1} \end{Schunk}

\hypertarget{computing-survival-probabilities-and-summary-statistics}{%
\subsection{Computing Survival Probabilities and Summary
Statistics}\label{computing-survival-probabilities-and-summary-statistics}}

While viewing plots such as those explained above are very useful in
Survival Analysis, they only tell half of the story.

\hypertarget{summary}{%
\subsection{Summary}\label{summary}}

This file is only a basic article template. For full details of
\emph{The R Journal} style and information on how to prepare your
article for submission, see the
\href{https://journal.r-project.org/share/author-guide.pdf}{Instructions
for Authors}.

\bibliography{RJreferences}


\address{%
Victor Wilson\\
California Polytechnic State University, San Luis Obispo - Statistics
Department\\
\\
}
\href{mailto:victorjw26@yahoo.com}{\nolinkurl{victorjw26@yahoo.com}}

\address{%
Ashley Jacobson\\
California Polytechnic State University, San Luis Obispo - Statistics
Department\\
\\
}
\href{mailto:ashleypjacobson@gmail.com}{\nolinkurl{ashleypjacobson@gmail.com}}

