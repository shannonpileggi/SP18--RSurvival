% !TeX root = RJwrapper.tex
\title{Introducing survfuncs Package - Simple Survival Analysis with R}
\author{by Victor Wilson, Ashley Jacobson}

\maketitle

\abstract{%
This article introduces the R package survfuncs, which executes
parametric survival analysis techniques similar to those in Minitab. The
functions available in this package carry out basic survival analysis
techniques. Among these are plotting hazard, cumulative hazard, and
survival curves, based on specified parametric distributions, computing
survival probabilites, and computing summary statistics. We describe
appropriate usage of these functions, what the output means, and provide
examples of how to utilizie this functions in real-world datasets.
}

% Any extra LaTeX you need in the preamble

\hypertarget{introduction}{%
\subsection{Introduction}\label{introduction}}

Survival analysis is a branch of statistics that primarily deals with
analyzing the time until an event of interest occurs. This event could
be a variety of different things such as death, development of disease,
or first score of a soccer game. Observations in survival analysis may
also be desrcibed as censored, which occurs when an observation's
survival time is incomplete. The most common way that this occurs is
through right censoring, which occurs when a subject does not experience
the event of interest within the duration of the study. Right censoring
can also occur if a subject drops out before the end of the study and
does not experience the event of interest. Due to the inherent issue of
censoring that is typically found in datasets involving survival
analysis, computations and analyses can be difficult to carry out with
many standard functions avaiable in R, as the majority of these do not
account for censored data. The censored data here is of value and we
cannot merely eliminate the observations which have censored data.

Some of the most popular techniques and statistics utilized when
carrying out a survival analysis are computing what are known as the
survival and hazard functions. The survival function is important
because it gives the proability of surviving (also known as not
experiencing the event of interest), for any given time. Similarly, the
hazard function is also useful to compute because it gives the
conditional probability that the subject will experience the event in
the next instance of time, given that they have survived up until the
specified point in time. Other popular statistics that are utilized are
median survial time, mean survival time, and percentiles of survival
time. In this package, all of the functions that we developed utilize
parametric methods of survival anlaysis, which assumes that the
distribution of the survival times follows a known probability
distribution.

Currently, R does have many survival packages that have been developed
that compute some of these statistics. However, we noticed that Minitab
has very concise and easy to utilize functions for computing and
displaying many of these survival statistics and plots, but this same
output is not readily available in any single one package in R, or in
some cases not available at all. Thus, we decided to develop a package
that emulates the output found in Minitab for survival analysis, which
contains all of these commonly utilized statistics and plots.

This paper describes the functions that this package contains, how the
data is formatted in order to utilize these functions, and what the
output of these functions represent. There are 3 major groups of
functions that we created: fitting the censored data, displaying plots
(hazard, cumulative hazard, and survival), and computing summary
statistics (mean, median, survival probabilities). The majority of this
paper will be organized following these groups of functions.

\hypertarget{fitting-right-censored-survival-data}{%
\subsection{Fitting Right Censored Survival
Data}\label{fitting-right-censored-survival-data}}

As mentoined previously, this function is very similar to the function
fitdistcens found in the
\href{https://cran.r-project.org/web/packages/fitdistrplus/index.html}{fitdistrplus}
package. This package works to help fit a parametric distribution to
data that is right censored. This function also computes the Maximum
Likelihood Estimate (MLE).

\hypertarget{example}{%
\subsubsection{Example}\label{example}}

\begin{Schunk}
\begin{Soutput}
#> Fitting of the distribution ' logis ' on censored data by maximum likelihood 
#> Parameters:
#>           estimate
#> location 133.74054
#> scale     20.48286
#> Fixed parameters:
#> data frame with 0 columns and 0 rows
\end{Soutput}
\end{Schunk}

\hypertarget{displaying-plots}{%
\subsection{Displaying Plots}\label{displaying-plots}}

\hypertarget{survival-plots}{%
\subsubsection{Survival Plots}\label{survival-plots}}

This section introduces an overview of the many types of plots that are
available to be displayed via this package. Some of the most common
plots used in Survival Analysis are survival plots, hazard plots, and
cumulative hazard plots. Survival plots are used to estimate the
proportion of subjects that survive beyond a specified time \textbf{t}.
We were motivated to create the function plot\_surv function in an
attempt to mimic the hazard plots that are available in Minitab. This
function plots the survival curve of right censored data given that it
follows a specified parametric distribution. Some examples of the
distributions that this function supports are the Weibull, Log-Normal,
Exponential, Normal, and Logistic distributions. This function also
provides the option to plot by a grouping variable, which if specified,
displays seperate survival plots for each group of the specified
variable.

\begin{Schunk}

\includegraphics{Simple_Survival_Analysis_files/figure-latex/unnamed-chunk-2-1} \end{Schunk}

In this example, we fit a Weibull distribution to the rats dataset
available in the
\href{https://cran.r-project.org/web/packages/survival/index.html}{survival}
package, grouping by the ``sex'' variable. As seen in the plot above,
two different survival curves were plotted. The blue line represents the
estimated survival curve for male rats, while the red line represents
the estimated survival curve for female rats.

\hypertarget{summary}{%
\subsection{Summary}\label{summary}}

This file is only a basic article template. For full details of
\emph{The R Journal} style and information on how to prepare your
article for submission, see the
\href{https://journal.r-project.org/share/author-guide.pdf}{Instructions
for Authors}.

\bibliography{RJreferences}


\address{%
Victor Wilson\\
California Polytechnic State University, San Luis Obispo - Statistics
Department\\
\\
}
\href{mailto:victorjw26@yahoo.com}{\nolinkurl{victorjw26@yahoo.com}}

\address{%
Ashley Jacobson\\
California Polytechnic State University, San Luis Obispo - Statistics
Department\\
\\
}
\href{mailto:ashleypjacobson@gmail.com}{\nolinkurl{ashleypjacobson@gmail.com}}

