% !TeX root = RJwrapper.tex
\title{Introducing parmsurvfit Package - Simple Parametric Survival Analysis
with R}
\author{by Victor Wilson, Ashley Jacobson, Shannon Pileggi}

\maketitle

\abstract{%
This article introduces the R package parmsurvfit, which executes basic
parametric survival analysis techniques similar to those in Minitab.
Among these are plotting hazard, cumulative hazard, survival, and
density curves, computing survival probabilites, and computing summary
statistics based on a specified parametric distribution. We describe
appropriate usage of these functions, interpretation of output, and
provide examples of how to utilize these functions in real-world
datasets.
}

% Any extra LaTeX you need in the preamble

\hypertarget{introduction}{%
\subsection{Introduction}\label{introduction}}

Survival analysis is a branch of statistics that primarily deals with
analyzing the time until an event of interest occurs. This event could
be a variety of different things such as death, development of disease,
or first score of a soccer game. Observations in survival analysis may
also be desrcibed as censored, which occurs when an observation's
survival time is incomplete. The most common way that this occurs is
through right censoring, which occurs when a subject does not experience
the event of interest within the duration of the study. Right censoring
can also occur if a subject drops out before the end of the study and
does not experience the event of interest. Due to the inherent issue of
censoring that is typically found in datasets involving survival
analysis, computations and analyses can be difficult to carry out with
many standard functions avaiable in R, as the majority of these do not
account for censored data. The censored data here is of value and we
cannot merely eliminate the observations which have censored data.

Some of the most popular techniques and statistics utilized when
carrying out a survival analysis are computing what are known as the
survival and hazard functions. The survival function is important
because it gives the proability of surviving (also known as not
experiencing the event of interest) beyond any given time \(t\).
Similarly, the hazard function is also useful to compute because it
gives the conditional probability that the subject will experience the
event in the next instance of time, given that they have survived up
until the specified point in time. Other popular statistics that are
utilized are median survial time, mean survival time, and percentiles of
survival time. In this package, all of the functions that we developed
utilize parametric methods of survival anlaysis, which assumes that the
distribution of the survival times follows a known probability
distribution.

Currently, R does have many survival packages that address non-parametic
survival analysis, such as the \pkg{survival} package. Moreover, R does
have some packages that aid in estimation for parametric survival
analysis, including \pkg{fitdistrplus}. However, Minitab has very
concise and easy to utilize functions for computing and displaying many
parametric survival statistics and plots, but this same output is not
readily available in any single one package in R, or in some cases not
available at all. Thus, we decided to develop a package that emulates
the output found in Minitab for parametric survival analysis, which
contains all of these commonly utilized statistics and plots.

This paper describes the functions that the \pkg{parmsurvfit} package
contains, how the data is formatted in order to utilize these functions,
and what the output of these functions represent. There are four major
groups of functions that we created: fitting the censored data,
displaying plots (density, hazard, cumulative hazard, and survival),
computing statistics (mean, median, survival probabilities), and
assessing fit (qqplot, Anderson Darling statistic). The majority of this
paper will be organized following these groups of functions.

\hypertarget{fitting-right-censored-survival-data}{%
\subsection{Fitting Right Censored Survival
Data}\label{fitting-right-censored-survival-data}}

As mentoined previously, this function is very similar to the function
fitdistcens found in the
\href{https://cran.r-project.org/web/packages/fitdistrplus/index.html}{fitdistrplus}
package, which computes the Maximum Likelihood Estimates (MLEs) for
right-censored data. \textbf{Is the data organized differently than
required for fitdistcens? Explain the output}

\hypertarget{example}{%
\subsubsection{Example}\label{example}}

\begin{Schunk}
\begin{Soutput}
#> Fitting of the distribution ' logis ' on censored data by maximum likelihood 
#> Parameters:
#>           estimate
#> location 16.741581
#> scale     2.798533
#> Fixed parameters:
#> data frame with 0 columns and 0 rows
\end{Soutput}
\end{Schunk}

\hypertarget{displaying-plots}{%
\subsection{Displaying Plots}\label{displaying-plots}}

This section introduces an overview of the many types of plots that are
available to be displayed via this package. Some of the most common
plots used in Survival Analysis are survival plots, hazard plots,
probability density plots, and cumulative hazard plots. We designed
these functions with an intent to have the output displayed be very easy
to read and interpret. Below is a list of each function and it's
relationship to other functions, as well as the forumla used to compute
each function.

\begin{tabular}{ll}
\hline
Function & Relationships  \\
\hline
PDF & ${f(t)=\frac{d}{dt}F(t)}$\\
CDF  & ${F(t)=\int_0^t f(y)dy}$\\
Survival & ${S(t)=1-F(t)=\exp[-H(t)]=\exp[-\int_0^th(y)dy]}$ \\
Hazard & ${h(t)=f(t)/S(t)=-\frac{d}{dt}\ln[S(t)]}$ \\
Cum. Haz. & ${H(t)=\int_0^t h(y)dy=-\ln[S(t)]}$\\
\hline
\end{tabular}

\hypertarget{density-plotshistograms}{%
\subsubsection{Density Plots/histograms}\label{density-plotshistograms}}

The \code{plot\_density} function creates a histogram of the data and
overlays the density function of a fitted parametric distribution.
Parameters estimates for the specified parametric distribution are
provided as well. This function also supports the ability to plot
seperate histograms and density functions for each level of a grouping
variable. An example of this function is shown below:

\begin{Schunk}
\begin{Sinput}
plot_density(Firstdrink, "norm", time = "Age", by = "Gender")
\end{Sinput}

\includegraphics{Simple_Survival_Analysis_files/figure-latex/unnamed-chunk-2-1} 
\includegraphics{Simple_Survival_Analysis_files/figure-latex/unnamed-chunk-2-2} 
\includegraphics{Simple_Survival_Analysis_files/figure-latex/unnamed-chunk-2-3} \end{Schunk}

We ran the \code{plot\_density} function, utilizing the
\file{Firstdrink.txt} dataset available in our package. This dataset
contains data on the age of first consumption of alcholic beverage for
1000 indiviuals. As seen above, a seperate histogram and density plot
was created for males and females.

\hypertarget{survival-plots}{%
\subsubsection{Survival Plots}\label{survival-plots}}

Survival plots are used to estimate the proportion of subjects that
survive beyond a specified time \textbf{t}. We were motivated to create
the function \code{plot\_surv} in an attempt to create hazard plots that
are easy to produce, when dealing with data set up for Survival
Analysis. This function plots the survival curve of right censored data
given that it follows a specified parametric distribution. Some examples
of the distributions that this function supports are the Weibull,
Log-Normal, Exponential, Normal, and Logistic distributions. This
function also provides the option to plot by a grouping variable, which
if specified, displays separate curves for each group of the specified
variable. In these plots, survival time is plotted on the x-axis, while
survival probability is plotted on the y-axis.

\begin{Schunk}

\includegraphics{Simple_Survival_Analysis_files/figure-latex/unnamed-chunk-3-1} \end{Schunk}

In this example, we fit a Weibull distribution to the Firstdrink
dataset, grouping by the Gender variable once again. As seen in the plot
above, two different survival curves were plotted. The blue line
represents the estimated survival curve for males, while the red line
represents the estimated survival curve for females. From this plot, we
see that the survival curve for male rats is consistently above the
survival curve for female rats throughout all points in time. Due to
this, we can conclude that females tend to experience their first drink
of alcohol before males do.

\hypertarget{hazard-plots}{%
\subsubsection{Hazard plots}\label{hazard-plots}}

Hazard plots, on the other hand, are used to display the conditional
risk that a subject will experience the event of interest in the next
instant of time, given that the subject has survived beyond a certain
amount of time. Essentially, the hazard function attempts to assess the
risk that an individual who has not yet experienced the event in the
very next small amount of time. For example, if we observe that a person
has survived for 17 years without first trying alcohol, the hazard
function would estimate the risk that the person will experience their
first drink of alcohol in the next short instant of time, based on the
fact that it has already survived 17 years. We created the
\code{plot\_haz} function in order to easily plot hazard functions given
that it follows a specified parametric distribution, with the option to
include a grouping variable.

\textbf{INSERT HAZARD FUNCTION FORMULA}

\begin{Schunk}

\includegraphics{Simple_Survival_Analysis_files/figure-latex/unnamed-chunk-4-1} \end{Schunk}

From this plot above, also using the Firstdrink dataset , we can see
that as females continue to survive, their risk of experiencing the
event of interest in the next instant of time dramatically increases.
Similarly, males also seem to have a greater risk of experiencing the
event of interest as they survive longer, but their risk is lower than
that of females.

\hypertarget{cumulative-hazard-plots}{%
\subsubsection{Cumulative Hazard Plots}\label{cumulative-hazard-plots}}

While hazard plots are usually useful in assessing a subject's risk of
experiencing the event of interest in the next moment of time, these
plots can be difficult to read and understand at times. Sometimes, the
changes in hazard are very subtle, making it difficult to describe
periods of increasing and decreasing risk. In order to accurately assess
how hazard rates change over time, we investigate the accumulation of
hazard rates over time, known as cumulative hazard. The cumulative
hazard function, denoted \textbf{H(t)}, is the accumulated risk of
experiencing an event up to time \textbf{t}. Since the cumulative hazard
function is an accumulation of rates, it is important to note that this
function is non-decreasing and is hardly ever remains constant by
nature. We developed the function \code{plot\_cumhaz} in order to easily
display cumulative hazard plots, given that the data follows a specified
parametric distribution. The functionality of this function is nearly
identical to that of \code{plot\_haz}, with the only distinction being
that it plots cumulative hazard curves instead of hazard curves.

\textbf{Insert brief interpretation}

\begin{Schunk}

\includegraphics{Simple_Survival_Analysis_files/figure-latex/unnamed-chunk-5-1} \end{Schunk}

As we can see, the cumulative hazard function is increasing for both
males and females. This means that the risk for experiencing the first
drink of alcohol for both males and females in the next instant of time
is increasing over time, given that they have survived beyond time
\textbf{t}.

\hypertarget{computing-survival-probabilities-and-summary-statistics}{%
\subsection{Computing Survival Probabilities and Summary
Statistics}\label{computing-survival-probabilities-and-summary-statistics}}

While viewing plots such as those explained above are very useful in
survival analysis, they only tell half of the story. In order to carry
out a complete survival analysis, we must also compute statistics in
order to supplement our plots. Some of the most common statistics
utilized in parametric survival analysis are survival probabilities and
typical summary statistics such as the mean, median, standard deviation,
and percentiles of survival time.

\hypertarget{computing-survival-probabilites}{%
\subsection{Computing Survival
Probabilites}\label{computing-survival-probabilites}}

Being able to compute survival probabilites is especially of interest
because it estimates the probability that a subject will not have
experienced the event of interest beyond a specified time \(t\). We
developed the function \code{surv\_prob} to compute probability of
survival beyond time \(t\), given that the data follows a specified
parametric distribution.

\begin{Schunk}
\begin{Sinput}
library(parmsurvfit)
surv_prob(Firstdrink, "lnorm", 30, time="Age")
\end{Sinput}
\begin{Soutput}
#> P(T > 30) = 0.05191602
\end{Soutput}
\end{Schunk}

As seen in the output from the function above, utilizing the Firstdrink
data set and fiting a log-normal parametric distribution to the data,
the estimated probability that a person survives for 30 years without
having their first drin of alcohol is roughly 5\%.

\hypertarget{computing-summary-statistics}{%
\subsubsection{Computing Summary
Statistics}\label{computing-summary-statistics}}

Another useful form of output that we believed would be useful to also
have in R is a table of summary statistics. Summmary statistics that are
typically included are the mean, standard deviation, median, and IQR.
The surv\_summary function that we developed estimates various summary
statistics, including mean, median, standard deviation, and percentiles
of survival time given that the data follows a specified parametric
distribution. This function also supports the option to provide seperate
summary statistics for each level of a grouping variable, if desired.

\begin{Schunk}
\begin{Sinput}
library(parmsurvfit)
surv_summary(Firstdrink, "lnorm", time="Age", by="Gender")
\end{Sinput}
\begin{Soutput}
#> 
#> 
#> For level = 1 
#> meanlog      2.743253
#> sdlog        0.3505774
#> Log Liklihood    -1370.309
#> AIC      2744.619
#> BIC      2752.885
#> Mean     16.52221
#> StDev        5.974932
#> First Quantile   12.26552
#> Median       15.53745
#> Third Quantile   19.68219
#> 
#> For level = 2 
#> meanlog      2.862891
#> sdlog        0.3674697
#> Log Liklihood    -1658.704
#> AIC      3321.408
#> BIC      3329.987
#> Mean     18.73528
#> StDev        7.123736
#> First Quantile   13.66772
#> Median       17.51209
#> Third Quantile   22.43778
\end{Soutput}
\end{Schunk}

As seen above, after specifying the grouping variable of sex, two
seperate tables were produced, one for males and one for females. We can
see that the mean log survival time for females is smaller than the mean
log survival time for males. The standard deviation of log survival time
for females was also much smaller than that of males.

\hypertarget{summary}{%
\subsection{Summary}\label{summary}}

This file is only a basic article template. For full details of
\emph{The R Journal} style and information on how to prepare your
article for submission, see the
\href{https://journal.r-project.org/share/author-guide.pdf}{Instructions
for Authors}.

\bibliography{RJreferences}


\address{%
Victor Wilson\\
California Polytechnic State University, San Luis Obispo - Statistics
Department\\
\\
}
\href{mailto:victorjw26@yahoo.com}{\nolinkurl{victorjw26@yahoo.com}}

\address{%
Ashley Jacobson\\
California Polytechnic State University, San Luis Obispo - Statistics
Department\\
\\
}
\href{mailto:ashleypjacobson@gmail.com}{\nolinkurl{ashleypjacobson@gmail.com}}

\address{%
Shannon Pileggi\\
California Polytechnic State University, San Luis Obispo - Statistics
Department\\
\\
}
\href{mailto:spileggi@calpoly.edu}{\nolinkurl{spileggi@calpoly.edu}}

