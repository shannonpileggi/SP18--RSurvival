% !TeX root = RJwrapper.tex
\title{Introducing parmsurvfit Package - Simple Parametric Survival Analysis
with R}
\author{by Victor Wilson, Ashley Jacobson, Shannon Pileggi}

\maketitle

\abstract{%
This article introduces the R package parmsurvfit, which executes basic
parametric survival analysis techniques similar to those in Minitab.
Among these are plotting hazard, cumulative hazard, survival, and
density curves, computing survival probabilites, and computing summary
statistics based on a specified parametric distribution. We describe
appropriate usage of these functions, interpretation of output, and
provide examples of how to utilize these functions in real-world
datasets.
}

% Any extra LaTeX you need in the preamble

\hypertarget{introduction}{%
\subsection{Introduction}\label{introduction}}

Survival analysis is a branch of statistics that primarily deals with
analyzing the time until an event of interest occurs. This event could
be a variety of different things such as death, development of disease,
or first score of a soccer game. Observations in survival analysis may
also be described as censored, which occurs when an observation's
survival time is incomplete. The most common way that this occurs is
through right censoring, which occurs when a subject does not experience
the event of interest within the duration of the study. Right censoring
can also occur if a subject drops out before the end of the study and
does not experience the event of interest. Due to the inherent issue of
censoring that is typically found in datasets involving survival
analysis, computations and analyses can be difficult to carry out with
many standard functions available in R, as the majority of these do not
account for censored data. The censored data collected is of value and
we cannot merely eliminate the observations which have censored data.

\textless{}\textless{}\textless{}\textless{}\textless{}\textless{}\textless{}
HEAD Some of the most popular techniques and statistics utilized when
carrying out a survival analysis are computing what are known as the
survival and hazard functions. The survival function is important
because it gives the probability of surviving (also known as not
experiencing the event of interest) beyond any given time \(t\).
Similarly, the hazard function is also useful to compute because it
gives the conditional probability that the subject will experience the
event in the next instance of time, given that they have survived up
until the specified point in time. Other popular statistics that are
utilized are median survival time, mean survival time, and percentiles
of survival time. In this package, all of the functions that we
developed utilize parametric methods of survival analysis, which assumes
that the distribution of the survival times follows a known probability
distribution. ======= Some of the most popular techniques and statistics
utilized when carrying out a survival analysis are computing what are
known as the survival and hazard functions \citep{Kleinbaum2012}. The
survival function is important because it gives the proability of
surviving (also known as not experiencing the event of interest) beyond
any given time \(t\). Similarly, the hazard function is also useful to
compute because it gives the conditional probability that the subject
will experience the event in the next instance of time, given that they
have survived up until the specified point in time. Statistics that are
commonly reported include median survial time, mean survival time, and
percentiles of survival time. In this package, all of the functions that
we developed utilize parametric methods of survival anlaysis, which
assumes that the distribution of the survival times follows a known
probability distribution.
\textgreater{}\textgreater{}\textgreater{}\textgreater{}\textgreater{}\textgreater{}\textgreater{}
fab9f94e6d9225cfa4ed2576d8985a391a7b604d

Currently, R does have many survival packages that address
non-parametric survival analysis, such as the \pkg{survival} package.
Moreover, R does have some packages that aid in estimation for
parametric survival analysis, including \pkg{fitdistrplus}. However,
Minitab has very concise and easy to utilize functions for computing and
displaying many parametric survival statistics and plots, but this same
output is not readily available in any single one package in R, or in
some cases not available at all. Thus, we decided to develop a package
that emulates the output found in Minitab for parametric survival
analysis, which contains all of these commonly utilized statistics and
plots.

This paper describes the functions that the \pkg{parmsurvfit} package
contains, how the data is formatted in order to utilize these functions,
and what the output of these functions represent. There are three major
groups of functions that we created: assessing fit, survival functions
(density, hazard, cumulative hazard, and survival), and computing
statistics (mean, median, survival probabilities). The majority of this
paper will be organized following these groups of functions.

\textless{}\textless{}\textless{}\textless{}\textless{}\textless{}\textless{}
HEAD \#\# Assessing fit Since all of the functions available in this
package assume that the survival data follows a\\
known parametric distribution, it is important to have a method to
analyze how well our assumed model fits the data. Utilizing such methods
will allow us choose a distribution that adequately fits the data. Some
common methods used to assess goodness of fit are viewing a histogram of
the data, Q-Q (Quantile-Quantile) plots, the Anderson-Darling Test.

\hypertarget{fitting-right-censored-survival-data}{%
\subsubsection{Fitting right censored survival
data}\label{fitting-right-censored-survival-data}}

As mentioned previously, this function is very similar to the function
\code{fitdistcens} found in the
\href{https://cran.r-project.org/web/packages/fitdistrplus/index.html}{fitdistrplus}
package, which computes the Maximum Likelihood Estimates (MLEs) for
right-censored data. The \code{fitdistcens} function requires data to be
organized into two columns, right and left. The right column indicates a
start time, and the left column indicates an end time. For example, if a
time is right censored, then the left column would contain the time, and
the right column would contain NA. This way of organizing survival data
allows for different types of censoring to be unambiguous, but since the
\pkg{parmsurvfit} package will only handle right censored data, the
\code{fit\_data} function makes it so that the user doesn't have to
reorganize data.

\hypertarget{this-function-takes-in-two-required-columns-a-time-column-and-a-censor-column.-the-time-column-contains-the-time-to-event-variable-while-the-censor-column-indicates-whether-right-censoring-is-present-0-corresponds-to-censored-data-and-1-corresponds-to-complete-data.-the-function-also-takes-in-an-optional-grouping-variable-which-fits-the-data-for-each-group-individually.-the-function-returns-an-object-of-class-and-if-theres-a-grouping-variable-it-returns-a-list-of-objects-of-class-.-several-of-the-other-functions-available-in-this-package-also-utilize-arguments-that-are-similar-to-the-ones-found-in-this-function.}{%
\section{\texorpdfstring{This function takes in two required columns, a
Time column and a Censor column. The time column contains the
time-to-event variable, while the censor column indicates whether right
censoring is present (0 corresponds to censored data and 1 corresponds
to complete data). The function also takes in an optional grouping
variable, which fits the data for each group individually. The function
returns an object of class \samp{fitdistcens}, and if there's a grouping
variable it returns a list of objects of class \samp{fitdistcens}.
Several of the other functions available in this package also utilize
arguments that are similar to the ones found in this
function.}{This function takes in two required columns, a Time column and a Censor column. The time column contains the time-to-event variable, while the censor column indicates whether right censoring is present (0 corresponds to censored data and 1 corresponds to complete data). The function also takes in an optional grouping variable, which fits the data for each group individually. The function returns an object of class , and if there's a grouping variable it returns a list of objects of class . Several of the other functions available in this package also utilize arguments that are similar to the ones found in this function.}}\label{this-function-takes-in-two-required-columns-a-time-column-and-a-censor-column.-the-time-column-contains-the-time-to-event-variable-while-the-censor-column-indicates-whether-right-censoring-is-present-0-corresponds-to-censored-data-and-1-corresponds-to-complete-data.-the-function-also-takes-in-an-optional-grouping-variable-which-fits-the-data-for-each-group-individually.-the-function-returns-an-object-of-class-and-if-theres-a-grouping-variable-it-returns-a-list-of-objects-of-class-.-several-of-the-other-functions-available-in-this-package-also-utilize-arguments-that-are-similar-to-the-ones-found-in-this-function.}}

Then the \code{fit\_data} function produces maximum likelihood estimates
for right censored data based on the input distribution. The
\code{fit\_data} function utilizes the \code{fitdistcens} function in
the
\href{https://cran.r-project.org/web/packages/fitdistrplus/index.html}{fitdistrplus}
package, but allows for more intuitive input of right-censored data than
as specified with \code{fitdistcens}. The \code{fit\_data} function in
the \pkg{parmsurvfit} package inputs two variables: \code{time} and
\code{censor}. The \code{time} variable contains the time-to-event
variable, while the \code{censor} variable indicates whether right
censoring is present (0 corresponds to censored data and 1 corresponds
to complete data). Furthermore, the user specifies the desired
parametric distribution in \code{dist} by inputting the base name of the
distribution as a character string. For example, to utilize the normal
distribution you would specify ``norm'' as it is the base of
\code{dnorm}, \code{pnorm}, etc. Commonly utilized distributions for
survival analysis include Weibull (``weibull''), log-normal (``lnorm''),
exponential (``exp''), and logistic (``logis'') distributions. The
function also takes in an optional grouping variable, which fits the
data for each group individually. The function returns an object of
class ``fitdistcens'', and if there's a grouping variable it returns a
list of objects of class ``fitdistcens''.

\begin{quote}
\begin{quote}
\begin{quote}
\begin{quote}
\begin{quote}
\begin{quote}
\begin{quote}
fab9f94e6d9225cfa4ed2576d8985a391a7b604d
\end{quote}
\end{quote}
\end{quote}
\end{quote}
\end{quote}
\end{quote}
\end{quote}

\hypertarget{example}{%
\subsubsection{Example}\label{example}}

In this example, we fit the weibull distribution to the
\file{firstdrink} data set where the time to event variable is age and
the variable that indicates censoring status is censor. The maximum
likelihood estimates of the location and scale parameters are returned.

\begin{Schunk}
\begin{Soutput}
#> Fitting of the distribution ' weibull ' on censored data by maximum likelihood 
#> Parameters:
#>        estimate
#> shape  2.536106
#> scale 19.684061
#> Fixed parameters:
#> data frame with 0 columns and 0 rows
\end{Soutput}
\end{Schunk}

\hypertarget{density-plotshistograms}{%
\subsubsection{Density plots/histograms}\label{density-plotshistograms}}

The \code{plot\_density} function creates a histogram of the data and
overlays the density function of a fitted parametric distribution.
Parameter estimates for the specified parametric distribution are
provided as well. This function also supports the ability to plot
separate histograms and density functions for each level of a grouping
variable. An example of this function is shown below:

\begin{Schunk}
\begin{Sinput}
plot_density(Firstdrink, "norm", time = "Age", by = "Gender")
\end{Sinput}

\includegraphics{Simple_Survival_Analysis_files/figure-latex/unnamed-chunk-2-1} 
\includegraphics{Simple_Survival_Analysis_files/figure-latex/unnamed-chunk-2-2} 
\includegraphics{Simple_Survival_Analysis_files/figure-latex/unnamed-chunk-2-3} \end{Schunk}

We ran the \code{plot\_density} function, utilizing the
\file{Firstdrink.txt} dataset available in our package. This dataset
contains data on the age of first consumption of an alcoholic beverage
for 1000 individuals. As seen above, a separate histogram and density
plot was created for males and females.

\hypertarget{q-q-plots}{%
\subsubsection{Q-Q plots}\label{q-q-plots}}

As mentioned before, Q-Q plots are a very popular method used to
evaluate the fit between two probability distributions. In these plots,
the hypothesized quantiles are plotted on one axis and the observed
quantiles are plotted on the other axis. A \emph{y=x} line is typically
included in these plots, because if the observed data fit the
hypothesized distribution perfectly, all of the points would lie exactly
on this line. We developed the \code{plot\_qqsurv} function to create a
quantile-quantile plot of right-censored data given that it follows a
specified distribution.

\begin{Schunk}
\begin{Sinput}
plot_qqsurv(Firstdrink, "exp", time="Age")
\end{Sinput}

\includegraphics{Simple_Survival_Analysis_files/figure-latex/unnamed-chunk-3-1} \end{Schunk}

We can create a Q-Q plot for the the Firstdrink data set to see how well
an Exponential distribution fits the data. As seen in the plot above,
there are some deviations from the provided \emph{y=x} line, indicates
that an Exponential distribution may not be an ideal fit for the data.

\hypertarget{anderson-darling-test}{%
\subsubsection{Anderson-Darling Test}\label{anderson-darling-test}}

While creating Q-Q plots are a great way to visualize how a particular
distribution may fit the data, it can be difficult at times to
definitively decide how well the plot fits the data. The
Anderson-Darling test provides a numerical test statistic that measures
how well the data fits a particular distribution. \textbf{CITE MINITAB
HELP PAGE}

\hypertarget{survival-functions}{%
\subsection{Survival Functions}\label{survival-functions}}

This section introduces an overview of the many types of survival
functions that are able to be displayed via this package. Some of the
most common functions used in survival analysis are the survival
function, the hazard function, and the cumulative hazard function.

\textless{}\textless{}\textless{}\textless{}\textless{}\textless{}\textless{}
HEAD We designed functions within this package to display a plot for
each of the aforementioned survival functions with an intent to have the
output displayed be very easy to read and interpret. Below is a list of
each function and it's relationship to other functions, as well as the
formula used to compute each function. ======= \#\# Plots for survival
analysis
\textgreater{}\textgreater{}\textgreater{}\textgreater{}\textgreater{}\textgreater{}\textgreater{}
fab9f94e6d9225cfa4ed2576d8985a391a7b604d

This section provides overview of the plots available in this package.
Some of the most common plots used in survival analysis are density
plots, survival plots, hazard plots, and cumulative hazard plots. We
designed these functions with an intent to have the output displayed be
very easy to read and interpret. Table \ref{table:functions} lists each
function and its relationship to other functions, as well as the forumla
used to compute each function.

\begin{table}
\begin{tabular}{ll}
\hline
Function & Relationships  \\
\hline
PDF & ${f(t)=\frac{d}{dt}F(t)}$\\
CDF  & ${F(t)=\int_0^t f(y)dy}$\\
Survival & ${S(t)=1-F(t)=\exp[-H(t)]=\exp[-\int_0^th(y)dy]}$ \\
Hazard & ${h(t)=f(t)/S(t)=-\frac{d}{dt}\ln[S(t)]}$ \\
Cum. Haz. & ${H(t)=\int_0^t h(y)dy=-\ln[S(t)]}$\\
\hline
\end{tabular}
\label{table:functions}
\end{table}

\textless{}\textless{}\textless{}\textless{}\textless{}\textless{}\textless{}
HEAD

\hypertarget{survival-plots}{%
\subsubsection{Survival plots}\label{survival-plots}}

\hypertarget{survival-plots-are-used-to-estimate-the-proportion-of-subjects-that-survive-beyond-a-specified-time-t.-we-were-motivated-to-create-the-function-in-an-attempt-to-create-hazard-plots-that-are-easy-to-produce-when-dealing-with-data-set-up-for-survival-analysis.-this-function-plots-the-survival-curve-of-right-censored-data-given-that-it-follows-a-specified-parametric-distribution.-some-examples-of-the-distributions-that-this-function-supports-are-the-weibull-log-normal-exponential-normal-and-logistic-distributions.-this-function-also-provides-the-option-to-plot-by-a-grouping-variable-which-if-specified-displays-separate-curves-for-each-group-of-the-specified-variable.-in-these-plots-survival-time-is-plotted-on-the-x-axis-while-survival-probability-is-plotted-on-the-y-axis.}{%
\section{\texorpdfstring{Survival plots are used to estimate the
proportion of subjects that survive beyond a specified time \textbf{t}.
We were motivated to create the function \code{plot\_surv} in an attempt
to create hazard plots that are easy to produce, when dealing with data
set up for survival analysis. This function plots the survival curve of
right censored data given that it follows a specified parametric
distribution. Some examples of the distributions that this function
supports are the Weibull, Log-Normal, Exponential, Normal, and Logistic
distributions. This function also provides the option to plot by a
grouping variable, which if specified, displays separate curves for each
group of the specified variable. In these plots, survival time is
plotted on the x-axis, while survival probability is plotted on the
y-axis.}{Survival plots are used to estimate the proportion of subjects that survive beyond a specified time t. We were motivated to create the function  in an attempt to create hazard plots that are easy to produce, when dealing with data set up for survival analysis. This function plots the survival curve of right censored data given that it follows a specified parametric distribution. Some examples of the distributions that this function supports are the Weibull, Log-Normal, Exponential, Normal, and Logistic distributions. This function also provides the option to plot by a grouping variable, which if specified, displays separate curves for each group of the specified variable. In these plots, survival time is plotted on the x-axis, while survival probability is plotted on the y-axis.}}\label{survival-plots-are-used-to-estimate-the-proportion-of-subjects-that-survive-beyond-a-specified-time-t.-we-were-motivated-to-create-the-function-in-an-attempt-to-create-hazard-plots-that-are-easy-to-produce-when-dealing-with-data-set-up-for-survival-analysis.-this-function-plots-the-survival-curve-of-right-censored-data-given-that-it-follows-a-specified-parametric-distribution.-some-examples-of-the-distributions-that-this-function-supports-are-the-weibull-log-normal-exponential-normal-and-logistic-distributions.-this-function-also-provides-the-option-to-plot-by-a-grouping-variable-which-if-specified-displays-separate-curves-for-each-group-of-the-specified-variable.-in-these-plots-survival-time-is-plotted-on-the-x-axis-while-survival-probability-is-plotted-on-the-y-axis.}}

\hypertarget{density-plotshistograms-1}{%
\subsubsection{Density
plots/histograms}\label{density-plotshistograms-1}}

The \code{plot\_density} function creates a histogram of the data and
overlays the density function of a fitted parametric distribution.
Parameters estimates for the specified parametric distribution are
provided as well. This function also supports the ability to plot
separate histograms and density functions for each level of a grouping
variable. Below, we fit the weibull distribution to age until first
drink by each value for gender using the \file{firstdrink} data set.
Three plots are produced, each based on separate MLE estimates: a plot
for males (level = 1), females (level = 2), and overall. In these plots,
all time to event data are plotted regardless of censoring status.

\begin{Schunk}
\begin{Sinput}
plot_density(data = Firstdrink, dist = "weibull", time = "Age", censor = "Censor", by = "Gender")
\end{Sinput}

\includegraphics{Simple_Survival_Analysis_files/figure-latex/unnamed-chunk-4-1} 
\includegraphics{Simple_Survival_Analysis_files/figure-latex/unnamed-chunk-4-2} 
\includegraphics{Simple_Survival_Analysis_files/figure-latex/unnamed-chunk-4-3} \end{Schunk}

\hypertarget{survival-plots-1}{%
\subsubsection{Survival plots}\label{survival-plots-1}}

The survival function \(S(t)\) estimates the proportion of subjects that
survive beyond a specified time \(t\). The \code{plot\_surv} function
plots the survival curve of right censored data given a specified
parametric distribution. This function also provides the option to plot
by a grouping variable, which if specified, displays separate curves for
each value of the specified grouping variable. In these plots, survival
time is plotted on the \(x\)-axis, while survival probability is plotted
on the \(y\)-axis.
\textgreater{}\textgreater{}\textgreater{}\textgreater{}\textgreater{}\textgreater{}\textgreater{}
fab9f94e6d9225cfa4ed2576d8985a391a7b604d

\begin{Schunk}

\includegraphics{Simple_Survival_Analysis_files/figure-latex/unnamed-chunk-5-1} \end{Schunk}

\textless{}\textless{}\textless{}\textless{}\textless{}\textless{}\textless{}
HEAD In this example, we fit a Weibull distribution to the `Firstdrink'
dataset, grouping by the Gender variable once again. As seen in the plot
above, two different survival curves were plotted. The blue line
represents the estimated survival curve for males, while the red line
represents the estimated survival curve for females. From this plot, we
see that the survival curve for females is consistently above the
survival curve for males throughout all points in time. Due to this, we
can conclude that males tend to experience their first drink of alcohol
before males do.

\hypertarget{hazard-plots}{%
\subsubsection{Hazard plots}\label{hazard-plots}}

\hypertarget{hazard-plots-on-the-other-hand-are-used-to-display-the-conditional-risk-that-a-subject-will-experience-the-event-of-interest-in-the-next-instant-of-time-given-that-the-subject-has-survived-beyond-a-certain-amount-of-time.-essentially-the-hazard-function-attempts-to-assess-the-risk-that-an-individual-who-has-not-yet-experienced-the-event-will-experience-the-event-in-the-very-next-small-amount-of-time.-for-example-if-we-observe-that-a-person-has-survived-for-17-years-without-first-trying-alcohol-the-hazard-function-would-estimate-the-risk-that-the-person-will-experience-their-first-drink-of-alcohol-in-the-next-short-instant-of-time-based-on-the-fact-that-they-have-already-survived-17-years.-we-created-the-function-in-order-to-easily-plot-hazard-functions-given-that-it-follows-a-specified-parametric-distribution-with-the-option-to-include-a-grouping-variable.}{%
\section{\texorpdfstring{Hazard plots, on the other hand, are used to
display the conditional risk that a subject will experience the event of
interest in the next instant of time, given that the subject has
survived beyond a certain amount of time. Essentially, the hazard
function attempts to assess the risk that an individual who has not yet
experienced the event, will experience the event in the very next small
amount of time. For example, if we observe that a person has survived
for 17 years without first trying alcohol, the hazard function would
estimate the risk that the person will experience their first drink of
alcohol in the next short instant of time, based on the fact that they
have already survived 17 years. We created the \code{plot\_haz} function
in order to easily plot hazard functions given that it follows a
specified parametric distribution, with the option to include a grouping
variable.}{Hazard plots, on the other hand, are used to display the conditional risk that a subject will experience the event of interest in the next instant of time, given that the subject has survived beyond a certain amount of time. Essentially, the hazard function attempts to assess the risk that an individual who has not yet experienced the event, will experience the event in the very next small amount of time. For example, if we observe that a person has survived for 17 years without first trying alcohol, the hazard function would estimate the risk that the person will experience their first drink of alcohol in the next short instant of time, based on the fact that they have already survived 17 years. We created the  function in order to easily plot hazard functions given that it follows a specified parametric distribution, with the option to include a grouping variable.}}\label{hazard-plots-on-the-other-hand-are-used-to-display-the-conditional-risk-that-a-subject-will-experience-the-event-of-interest-in-the-next-instant-of-time-given-that-the-subject-has-survived-beyond-a-certain-amount-of-time.-essentially-the-hazard-function-attempts-to-assess-the-risk-that-an-individual-who-has-not-yet-experienced-the-event-will-experience-the-event-in-the-very-next-small-amount-of-time.-for-example-if-we-observe-that-a-person-has-survived-for-17-years-without-first-trying-alcohol-the-hazard-function-would-estimate-the-risk-that-the-person-will-experience-their-first-drink-of-alcohol-in-the-next-short-instant-of-time-based-on-the-fact-that-they-have-already-survived-17-years.-we-created-the-function-in-order-to-easily-plot-hazard-functions-given-that-it-follows-a-specified-parametric-distribution-with-the-option-to-include-a-grouping-variable.}}

In this example, we fit a Weibull distribution to the \file{firstdrink}
dataset, grouping by the gender variable. As seen in the plot above, two
different survival curves were plotted. The red line represents the
estimated survival curve for males (gender = 1), while the blue line
represents the estimated survival curve for females (gender = 2). From
this plot, we see that the survival curve for females is consistently
above the survival curve for males throughout all points in time. Due to
this, we can conclude that males tend to experience their first drink of
alcohol before females do.

\hypertarget{hazard-plots-1}{%
\subsubsection{Hazard plots}\label{hazard-plots-1}}

Hazard plots, denoted \(h(t)\), are used to display the conditional risk
that a subject will experience the event of interest in the next instant
of time, given that the subject has survived beyond a certain time
\(t\). For example, if we observe that a person has survived for 17
years without first trying alcohol, the hazard function would estimate
the risk that the person will experience their first drink of alcohol in
the next short instant of time, based on the fact that the person has
already survived 17 years alcohol free. Hazard is not probability, and
therefore can take values greater than one. Moreover, the hazard
function can be both increasing or decreasing. The \code{plot\_haz}
function plots the hazard function based on a specified parametric
distribution, with the option to include a grouping variable.
\textgreater{}\textgreater{}\textgreater{}\textgreater{}\textgreater{}\textgreater{}\textgreater{}
fab9f94e6d9225cfa4ed2576d8985a391a7b604d

\begin{Schunk}

\includegraphics{Simple_Survival_Analysis_files/figure-latex/unnamed-chunk-6-1} \end{Schunk}

\textless{}\textless{}\textless{}\textless{}\textless{}\textless{}\textless{}
HEAD From this plot above, also using the `Firstdrink' dataset, we can
see that as males continue to survive, their risk of experiencing the
event of interest in the next instant of time dramatically increases.
Similarly, females also seem to have a greater risk of experiencing the
event of interest as they survive longer, but their risk is lower than
that of males.

\hypertarget{cumulative-hazard-plots}{%
\subsubsection{Cumulative hazard plots}\label{cumulative-hazard-plots}}

While hazard plots are usually useful in assessing a subject's risk of
experiencing the event of interest in the next moment of time, these
plots can be difficult to read and understand at times. Sometimes, the
changes in hazard are very subtle, making it difficult to describe
periods of increasing and decreasing risk. In order to accurately assess
how hazard rates change over time, we investigate the accumulation of
hazard rates over time, known as cumulative hazard. The cumulative
hazard function, denoted \textbf{H(t)}, is the accumulated risk of
experiencing an event up to time \emph{t}.

\hypertarget{since-the-cumulative-hazard-function-is-an-accumulation-of-rates-it-is-important-to-note-that-this-function-is-non-decreasing-and-hardly-ever-remains-constant-by-nature.-we-developed-the-function-in-order-to-easily-display-cumulative-hazard-plots-given-that-the-data-follows-a-specified-parametric-distribution.-the-functionality-of-this-function-is-nearly-identical-to-that-of-with-the-only-distinction-being-that-it-plots-cumulative-hazard-curves-instead-of-hazard-curves.}{%
\section{\texorpdfstring{Since the cumulative hazard function is an
accumulation of rates, it is important to note that this function is
non-decreasing and hardly ever remains constant by nature. We developed
the function \code{plot\_cumhaz} in order to easily display cumulative
hazard plots, given that the data follows a specified parametric
distribution. The functionality of this function is nearly identical to
that of \code{plot\_haz}, with the only distinction being that it plots
cumulative hazard curves instead of hazard
curves.}{Since the cumulative hazard function is an accumulation of rates, it is important to note that this function is non-decreasing and hardly ever remains constant by nature. We developed the function  in order to easily display cumulative hazard plots, given that the data follows a specified parametric distribution. The functionality of this function is nearly identical to that of , with the only distinction being that it plots cumulative hazard curves instead of hazard curves.}}\label{since-the-cumulative-hazard-function-is-an-accumulation-of-rates-it-is-important-to-note-that-this-function-is-non-decreasing-and-hardly-ever-remains-constant-by-nature.-we-developed-the-function-in-order-to-easily-display-cumulative-hazard-plots-given-that-the-data-follows-a-specified-parametric-distribution.-the-functionality-of-this-function-is-nearly-identical-to-that-of-with-the-only-distinction-being-that-it-plots-cumulative-hazard-curves-instead-of-hazard-curves.}}

From this plot above, also using the \code{firstdrink} data set, we can
see that as males continue to survive, their risk of experiencing the
event of interest in the next instant of time increases more
dramatically as compared to females. While females also seem to have a
greater risk of experiencing the event of interest as they survive
longer, but their risk is lower than that of males.

\hypertarget{cumulative-hazard-plots-1}{%
\subsubsection{Cumulative hazard
plots}\label{cumulative-hazard-plots-1}}

While hazard plots are usually useful in assessing a subject's risk of
experiencing the event of interest in the next moment of time, these
plots can be difficult to read and understand at times. Sometimes, the
changes in hazard are very subtle, making it difficult to describe
periods of increasing and decreasing risk. In order to accurately assess
how hazard rates change over time, we investigate the accumulation of
hazard rates over time, known as cumulative hazard. The cumulative
hazard function, denoted \(H(t)\), is the total accumulated risk of
experiencing an event up to time \(t\).

Since the cumulative hazard function is an accumulation of rates, it is
important to note that this function is non-decreasing and is hardly
ever remains constant by nature. The \code{plot\_cumhaz} function
displays cumulative hazard plots, given that the data follows a
specified parametric distribution. The functionality of this function is
nearly identical to that of \code{plot\_haz}, with the only distinction
being that it plots cumulative hazard curves instead of hazard curves.
\textgreater{}\textgreater{}\textgreater{}\textgreater{}\textgreater{}\textgreater{}\textgreater{}
fab9f94e6d9225cfa4ed2576d8985a391a7b604d

\begin{Schunk}

\includegraphics{Simple_Survival_Analysis_files/figure-latex/unnamed-chunk-7-1} \end{Schunk}

As expected, the cumulative hazard function is increasing for both males
and females. Here, the total accumulated risk of experiencing the first
drink of alcohol is greater for males compared to females

\textless{}\textless{}\textless{}\textless{}\textless{}\textless{}\textless{}
HEAD \#\# Statistics/Computations While viewing plots such as those
explained above are very useful in survival analysis, they only tell
half of the story. In order to carry out a complete survival analysis,
we must also compute statistics in order to supplement our plots. Some
of the most common statistics utilized in parametric survival analysis
are survival probabilities and typical summary statistics such as the
mean, median, standard deviation, and percentiles of survival time.

\hypertarget{computing-survival-probabilites}{%
\subsubsection{Computing survival
probabilites}\label{computing-survival-probabilites}}

\hypertarget{being-able-to-compute-survival-probabilities-is-especially-of-interest-because-it-estimates-the-probability-that-a-subject-will-not-have-experienced-the-event-of-interest-beyond-a-specified-time-t.-we-developed-the-function-to-compute-probability-of-survival-beyond-time-t-given-that-the-data-follows-a-specified-parametric-distribution.}{%
\section{\texorpdfstring{Being able to compute survival probabilities is
especially of interest because it estimates the probability that a
subject will not have experienced the event of interest beyond a
specified time \(t\). We developed the function \code{surv\_prob} to
compute probability of survival beyond time \(t\), given that the data
follows a specified parametric
distribution.}{Being able to compute survival probabilities is especially of interest because it estimates the probability that a subject will not have experienced the event of interest beyond a specified time t. We developed the function  to compute probability of survival beyond time t, given that the data follows a specified parametric distribution.}}\label{being-able-to-compute-survival-probabilities-is-especially-of-interest-because-it-estimates-the-probability-that-a-subject-will-not-have-experienced-the-event-of-interest-beyond-a-specified-time-t.-we-developed-the-function-to-compute-probability-of-survival-beyond-time-t-given-that-the-data-follows-a-specified-parametric-distribution.}}

\hypertarget{computing-survival-probabilities-and-summary-statistics}{%
\subsection{Computing survival probabilities and summary
statistics}\label{computing-survival-probabilities-and-summary-statistics}}

Some common statistics utilized in parametric survival analysis include
survival probabilities and typical summary statistics such as the mean,
median, standard deviation, and percentiles of survival time.

\hypertarget{computing-survival-probabilites-1}{%
\subsubsection{Computing survival
probabilites}\label{computing-survival-probabilites-1}}

A survival probabily estimates the probability that a subject will not
have experienced the event of interest beyond a specified time \(t\). We
developed the function \code{surv\_prob} to compute probability of
survival beyond time \(t\), given that the data follows a specified
parametric distribution.
\textgreater{}\textgreater{}\textgreater{}\textgreater{}\textgreater{}\textgreater{}\textgreater{}
fab9f94e6d9225cfa4ed2576d8985a391a7b604d

\begin{Schunk}
\begin{Sinput}
surv_prob(data = Firstdrink, dist = "weibull", num = 30, time = "Age", censor = "Censor")
\end{Sinput}
\begin{Soutput}
#> P(T > 30) = 0.05439142
\end{Soutput}
\end{Schunk}

\textless{}\textless{}\textless{}\textless{}\textless{}\textless{}\textless{}
HEAD As seen in the output from the function above, utilizing the
`Firstdrink' data set and fitting a log-normal parametric distribution
to the data, the estimated probability that a person survives for 30
years without having their first drink of alcohol is roughly 5\%.

\hypertarget{computing-summary-statistics}{%
\subsubsection{Computing summary
statistics}\label{computing-summary-statistics}}

\hypertarget{another-useful-form-of-output-that-we-believed-would-be-useful-to-also-have-in-r-is-a-table-of-summary-statistics.-summary-statistics-that-are-typically-included-are-the-mean-standard-deviation-median-and-iqr.-the-function-that-we-developed-estimates-various-summary-statistics-including-mean-median-standard-deviation-and-percentiles-of-survival-time-given-that-the-data-follows-a-specified-parametric-distribution.-this-function-also-supports-the-option-to-provide-separate-summary-statistics-for-each-level-of-a-grouping-variable-if-desired.}{%
\section{\texorpdfstring{Another useful form of output that we believed
would be useful to also have in R is a table of summary statistics.
Summary statistics that are typically included are the mean, standard
deviation, median, and IQR. The \code{surv_summary} function that we
developed estimates various summary statistics, including mean, median,
standard deviation, and percentiles of survival time given that the data
follows a specified parametric distribution. This function also supports
the option to provide separate summary statistics for each level of a
grouping variable, if
desired.}{Another useful form of output that we believed would be useful to also have in R is a table of summary statistics. Summary statistics that are typically included are the mean, standard deviation, median, and IQR. The  function that we developed estimates various summary statistics, including mean, median, standard deviation, and percentiles of survival time given that the data follows a specified parametric distribution. This function also supports the option to provide separate summary statistics for each level of a grouping variable, if desired.}}\label{another-useful-form-of-output-that-we-believed-would-be-useful-to-also-have-in-r-is-a-table-of-summary-statistics.-summary-statistics-that-are-typically-included-are-the-mean-standard-deviation-median-and-iqr.-the-function-that-we-developed-estimates-various-summary-statistics-including-mean-median-standard-deviation-and-percentiles-of-survival-time-given-that-the-data-follows-a-specified-parametric-distribution.-this-function-also-supports-the-option-to-provide-separate-summary-statistics-for-each-level-of-a-grouping-variable-if-desired.}}

Utilizing the firstdrink data set and fiting a weibull distribution to
the data, the estimated probability that a person survives for 30 years
without having their first drink of alcohol is roughly 5\%.

\hypertarget{computing-summary-statistics-1}{%
\subsubsection{Computing summary
statistics}\label{computing-summary-statistics-1}}

The \code{surv_summary} function estimates various summary statistics,
including mean, median, standard deviation, and percentiles of survival
time given that the data follows a specified parametric distribution.
This function also supports the option to provide seperate summary
statistics for each level of a grouping variable, if desired.\\
\textgreater{}\textgreater{}\textgreater{}\textgreater{}\textgreater{}\textgreater{}\textgreater{}
fab9f94e6d9225cfa4ed2576d8985a391a7b604d

\begin{Schunk}
\begin{Sinput}
surv_summary(data = Firstdrink, dist = "weibull", time = "Age", censor = "Censor", by = "Gender")
\end{Sinput}
\begin{Soutput}
#> 
#> 
#> For level = 1 
#> shape        2.637645
#> scale        18.2804
#> Log Liklihood    -1425.271
#> AIC      2854.541
#> BIC      2862.808
#> Mean     16.24398
#> StDev        6.625303
#> First Quantile   11.39844
#> Median       15.90884
#> Third Quantile   20.6903
#> 
#> For level = 2 
#> shape        2.516025
#> scale        20.85053
#> Log Liklihood    -1730.273
#> AIC      3464.546
#> BIC      3473.126
#> Mean     18.50288
#> StDev        7.872356
#> First Quantile   12.70752
#> Median       18.02407
#> Third Quantile   23.74094
\end{Soutput}
\end{Schunk}

\textless{}\textless{}\textless{}\textless{}\textless{}\textless{}\textless{}
HEAD As seen above, after specifying the grouping variable of sex, two
separate tables were produced, one for males and one for females. We can
see that the mean log survival time for males is smaller than the mean
log survival time for females. The standard deviation of log survival
time for males was also much smaller than that of females. ======= As
seen above, after specifying the grouping variable of gender, two
separate tables were produced, one for males and one for females. We can
see that the mean survival time for males (16.2 years) is less than the
mean survival time for females (18.5 years).

\hypertarget{assessing-fit}{%
\subsection{Assessing fit}\label{assessing-fit}}

Since all of the functions available in this package assume that the
survival data follows a known parametric distribution, it is important
to have a method to analyze how well our assumed model fits the data.
Utilizing such methods will allow us choose a distribution that
adequately fits the data. Two common methods used to assess goodness of
fit are Q-Q (Quantile-Quantile) plots and the Anderson-Darling test
statistic.

\hypertarget{q-q-plots-1}{%
\subsubsection{Q-Q plots}\label{q-q-plots-1}}

In Q-Q plots the hypothesized quantiles are plotted on the \(y\)-axis
and the observed quantiles are plotted on the \(x\)-axis. A \(y=x\) line
is typically included in these plots, because if the observed data fit
the hypothesized distribution perfectly, all of the points would lie
exactly on this diagonal line. The \code{plot\_qqsurv} function creates
a quantile-quantile plot of right-censored data given that it follows a
specified distribution.

\textbf{May need to modify: The points are plotted according to the
median rank method \citep{Minitabqq}.}

\begin{Schunk}
\begin{Sinput}
plot_qqsurv(data = Firstdrink, dist = "weibull", time = "Age", censor = "Censor")
\end{Sinput}

\includegraphics{Simple_Survival_Analysis_files/figure-latex/unnamed-chunk-10-1} \end{Schunk}

The plot above displays a Q-Q plot for the the \file{firstdrink} data
set to see how well the Weibull distribution fits the data. As seen in
the plot above, there are some deviations from the provided \(y=x\)
line, indicates that an Weibull distribution may not be an ideal fit for
the data.

\hypertarget{anderson-darling-test-statistic}{%
\subsubsection{Anderson-Darling test
statistic}\label{anderson-darling-test-statistic}}

While creating Q-Q plots are a great way to visualize how a particular
distribution may fit the data, it can be difficult at times to
definitively decide how well the plot fits the data. The
Anderson-Darling test statistic provides a numerical test statistic that
measures how well the data fits a particular distribution, such that
lower values indicate a better fit. Computation of the test statistic
adhered to Minitab's documentation, utilizing the median rank plotting
method \citep{Minitabgof}.
\textgreater{}\textgreater{}\textgreater{}\textgreater{}\textgreater{}\textgreater{}\textgreater{}
fab9f94e6d9225cfa4ed2576d8985a391a7b604d

\hypertarget{conclusion}{%
\subsection{Conclusion}\label{conclusion}}

\textless{}\textless{}\textless{}\textless{}\textless{}\textless{}\textless{}
HEAD The R package \pkg{parmsurvfit} allows for parametric survival
analysis methods involving right-censored data to be easily computed in
R. The overall goal of developing this package was to provide a central
package for R users to find typical methods used in Survival Analysis
such as computing survival probabilities, create survival and hazard
plots, and assess goodness of fit of a parametric distribution fit to a
dataset.

\bibliography{Simple Survival Analysis}

=======

The R package \pkg{parmsurvfit} allows for simple parametric survial
analysis methods involving right-censored data to be easily computed in
R. The overall goal of developing this package was to provide a central
package for R users to utilize typical parametric survival analysis
methods such as computing survival probabilities, creating survival and
hazard plots, and assessing goodness of fit of a parametric distribution
fit to a dataset with right-censored observations.

\hypertarget{acknowledgments}{%
\subsection{Acknowledgments}\label{acknowledgments}}

This package development was funded by the Bill and Linda Frost fund at
California Polytechnic State University, San Luis Obispo. We would also
like to thank Jeff Sklar for his input and guidance.

\bibliography{RJreferences}

\begin{quote}
\begin{quote}
\begin{quote}
\begin{quote}
\begin{quote}
\begin{quote}
\begin{quote}
fab9f94e6d9225cfa4ed2576d8985a391a7b604d
\end{quote}
\end{quote}
\end{quote}
\end{quote}
\end{quote}
\end{quote}
\end{quote}


\address{%
Victor Wilson\\
California Polytechnic State University, San Luis Obispo - Statistics
Department\\
\\
}
\href{mailto:victorjw26@yahoo.com}{\nolinkurl{victorjw26@yahoo.com}}

\address{%
Ashley Jacobson\\
California Polytechnic State University, San Luis Obispo - Statistics
Department\\
\\
}
\href{mailto:ashleypjacobson@gmail.com}{\nolinkurl{ashleypjacobson@gmail.com}}

\address{%
Shannon Pileggi\\
California Polytechnic State University, San Luis Obispo - Statistics
Department\\
\\
}
\href{mailto:spileggi@calpoly.edu}{\nolinkurl{spileggi@calpoly.edu}}

