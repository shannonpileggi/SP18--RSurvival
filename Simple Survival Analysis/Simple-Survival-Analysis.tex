% !TeX root = RJwrapper.tex
\title{Introducing parmsurvfit Package - Simple Parametric Survival Analysis
with R}
\author{by Victor Wilson, Ashley Jacobson, Shannon Pileggi}

\maketitle

\abstract{%
This article introduces the R package parmsurvfit, which executes basic
parametric survival analysis techniques similar to those in Minitab.
Among these are plotting hazard, cumulative hazard, survival, and
density curves, computing survival probabilites, and computing summary
statistics based on a specified parametric distribution. We describe
appropriate usage of these functions, interpretation of output, and
provide examples of how to utilize these functions in real-world
datasets.
}

% Any extra LaTeX you need in the preamble

\hypertarget{introduction}{%
\subsection{Introduction}\label{introduction}}

Survival analysis is a branch of statistics that primarily deals with
analyzing the time until an event of interest occurs. This event could
be a variety of different things such as death, development of disease,
or first score of a soccer game. Observations in survival analysis may
also be desrcibed as censored, which occurs when an observation's
survival time is incomplete. The most common way that this occurs is
through right censoring, which occurs when a subject does not experience
the event of interest within the duration of the study. Right censoring
can also occur if a subject drops out before the end of the study and
does not experience the event of interest. Due to the inherent issue of
censoring that is typically found in datasets involving survival
analysis, computations and analyses can be difficult to carry out with
many standard functions avaiable in R, as the majority of these do not
account for censored data. The censored data here is of value and we
cannot merely eliminate the observations which have censored data.

Some of the most popular techniques and statistics utilized when
carrying out a survival analysis are computing what are known as the
survival and hazard functions. The survival function is important
because it gives the proability of surviving (also known as not
experiencing the event of interest) beyond any given time \(t\).
Similarly, the hazard function is also useful to compute because it
gives the conditional probability that the subject will experience the
event in the next instance of time, given that they have survived up
until the specified point in time. Other popular statistics that are
utilized are median survial time, mean survival time, and percentiles of
survival time. In this package, all of the functions that we developed
utilize parametric methods of survival anlaysis, which assumes that the
distribution of the survival times follows a known probability
distribution.

Currently, R does have many survival packages that address non-parametic
survival analysis, such as the \pkg{survival} package. Moreover, R does
have some packages that aid in estimation for parametric survival
analysis, including \pkg{fitdistrplus}. However, Minitab has very
concise and easy to utilize functions for computing and displaying many
parametric survival statistics and plots, but this same output is not
readily available in any single one package in R, or in some cases not
available at all. Thus, we decided to develop a package that emulates
the output found in Minitab for parametric survival analysis, which
contains all of these commonly utilized statistics and plots.

This paper describes the functions that the \pkg{parmsurvfit} package
contains, how the data is formatted in order to utilize these functions,
and what the output of these functions represent. There are four major
groups of functions that we created: fitting the censored data,
displaying plots (density, hazard, cumulative hazard, and survival),
computing statistics (mean, median, survival probabilities), and
assessing fit (qqplot, Anderson Darling statistic). The majority of this
paper will be organized following these groups of functions.

\hypertarget{fitting-right-censored-survival-data}{%
\subsection{Fitting Right Censored Survival
Data}\label{fitting-right-censored-survival-data}}

As mentoined previously, this function is very similar to the function
fitdistcens found in the
\href{https://cran.r-project.org/web/packages/fitdistrplus/index.html}{fitdistrplus}
package, which computes the Maximum Likelihood Estimates (MLEs) for
right-censored data. \textbf{Is the data organized differently than
required for fitdistcens? Explain the output}

\hypertarget{example}{%
\subsubsection{Example}\label{example}}

\begin{Schunk}
\begin{Soutput}
#> Fitting of the distribution ' logis ' on censored data by maximum likelihood 
#> Parameters:
#>           estimate
#> location 16.741581
#> scale     2.798533
#> Fixed parameters:
#> data frame with 0 columns and 0 rows
\end{Soutput}
\end{Schunk}

\hypertarget{displaying-plots}{%
\subsection{Displaying Plots}\label{displaying-plots}}

This section introduces an overview of the many types of plots that are
available to be displayed via this package. Some of the most common
plots used in Survival Analysis are survival plots, hazard plots,
probability density plots, and cumulative hazard plots. We designed
these functions with an intent to have the output displayed be very easy
to read and interpret. Below is a list of each function and it's
relationship to other functions, as well as the forumla used to compute
each function.

\begin{tabular}{ll}
\hline
Function & Relationships  \\
\hline
PDF & ${f(t)=\frac{d}{dt}F(t)}$\\
CDF  & ${F(t)=\int_0^t f(y)dy}$\\
Survival & ${S(t)=1-F(t)=\exp[-H(t)]=\exp[-\int_0^th(y)dy]}$ \\
Hazard & ${h(t)=f(t)/S(t)=-\frac{d}{dt}\ln[S(t)]}$ \\
Cum. Haz. & ${H(t)=\int_0^t h(y)dy=-\ln[S(t)]}$\\
\hline
\end{tabular}

\hypertarget{density-plotshistograms}{%
\subsubsection{Density Plots/histograms}\label{density-plotshistograms}}

The \code{plot\_density} function creates a histogram of the data and
overlays the density function of a fitted parametric distribution.
Parameters estimates for the specified parametric distribution are
provided as well. This function also supports the ability to plot
seperate histograms and density functions for each level of a grouping
variable. An example of this function is shown below:

\begin{Schunk}
\begin{Sinput}
plot_density(Firstdrink, "norm", time = "Age", by = "Gender")
\end{Sinput}

\includegraphics{Simple_Survival_Analysis_files/figure-latex/unnamed-chunk-2-1} 
\includegraphics{Simple_Survival_Analysis_files/figure-latex/unnamed-chunk-2-2} 
\includegraphics{Simple_Survival_Analysis_files/figure-latex/unnamed-chunk-2-3} \end{Schunk}

We ran the \code{plot\_density} function, utilizing the
\file{Firstdrink.txt} dataset available in our package. This dataset
contains data on the age of first consumption of alcholic beverage for
1000 indiviuals. As seen above, a seperate histogram and density plot
was created for males and females.

\hypertarget{survival-plots}{%
\subsubsection{Survival Plots}\label{survival-plots}}

Survival plots are used to estimate the proportion of subjects that
survive beyond a specified time \textbf{t}. We were motivated to create
the function \code{plot\_surv} in an attempt to create hazard plots that
are easy to produce, when dealing with data set up for Survival
Analysis. This function plots the survival curve of right censored data
given that it follows a specified parametric distribution. Some examples
of the distributions that this function supports are the Weibull,
Log-Normal, Exponential, Normal, and Logistic distributions. This
function also provides the option to plot by a grouping variable, which
if specified, displays separate curves for each group of the specified
variable.

\begin{Schunk}

\includegraphics{Simple_Survival_Analysis_files/figure-latex/unnamed-chunk-3-1} \end{Schunk}

In this example, we fit a Weibull distribution to the rats dataset
available in the \CRANpkg{survival} package, grouping by the ``sex''
variable. The rats dataset contains 300 observations, with 3 rats each
being selected from 100 litters and 1 rat in each litter being
adminstered a drug. The event of interest in this study was whether or
not a rat developed a tumor following the beginning of the study. For
each rat, the litter number (1-100), type of treatment received (coded
as 1 = drug, 0 = control), time until development of tumor or last
follow-up (measured in days), final event status (1 = tumor, 0 =
censored), and sex were recorded. Below is an excerpt of the data frame.

\begin{Schunk}
\begin{Sinput}
library(parmsurvfit)
data(rats)
head(rats)
\end{Sinput}
\begin{Soutput}
#>   litter rx time status sex
#> 1      1  1  101      0   f
#> 2      1  0   49      1   f
#> 3      1  0  104      0   f
#> 4      2  1   91      0   m
#> 5      2  0  104      0   m
#> 6      2  0  102      0   m
\end{Soutput}
\end{Schunk}

For example, the rat represented in line 2 came from Litter 1, did not
receive the drug, experienced the development of a tumor at 49 days, and
was a female rat.

As seen in the plot above, two different survival curves were plotted.
The blue line represents the estimated survival curve for male rats,
while the red line represents the estimated survival curve for female
rats. From this plot, we see that the survival curve for male rats is
consistently above the survival curve for female rats throughout all
points in time. Due to this, we can conclude that female rats tend to
experience the event of interest much more quickly than male rats. This
can also be interpreted as male rats tend to survive longer than female
rats in this study.

\hypertarget{hazard-plots}{%
\subsubsection{Hazard plots}\label{hazard-plots}}

Hazard plots, on the other hand, are used to display the conditional
risk that a subject will experience the event of interest in the next
instant of time, given that the subject has survived beyond a certain
amount of time. Essentially, the hazard function attempts to assess the
risk that an individual who has not yet experienced the event in the
very next small amount of time. For example, if we observe that a rat
has survived for 75 days already, the hazard function would estimate the
risk that the rat will die in the next short instant of time, based on
the fact that it has already survived 75 days. We created the
\code{plot\_haz} function in order to easily plot hazard functions given
that it follows a specified parametric distribution, with the option to
include a grouping variable.

\textbf{INSERT HAZARD FUNCTION FORMULA}

\begin{Schunk}

\includegraphics{Simple_Survival_Analysis_files/figure-latex/unnamed-chunk-5-1} \end{Schunk}

From this plot above, also using the rats dataset available in the
survival package, we can see that as female rats continue to survive,
their risk of experiencing the event of interest in the next instant of
time dramatically increases. Contrastly, male rats do not seem to have a
greater risk of experiencing the event of interest as they survive
longer. This is demonstrated by the blue line being mostly flat across
all points of time.

\hypertarget{cumulative-hazard-plots}{%
\subsubsection{Cumulative Hazard Plots}\label{cumulative-hazard-plots}}

While hazard plots are usually useful in assessing a subject's risk of
experiencing the event of interest in the next moment of time, these
plots can be difficult to read and understand at times. Sometimes, the
changes in hazard are very subtle, making it difficult to describe
periods of increasing and decreasing risk. In order to accurately assess
how hazard rates change over time, we investigate the accumulation of
hazard rates over time, known as cumulative hazard. The cumulative
hazard function, denoted \textbf{H(t)}, is the accumulated risk of
experiencing an event up to time \textbf{t}. Since the cumulative hazard
function is an accumulation of rates, it is important to note that this
function is non-decreasing and is hardly ever remains constant by
nature. We developed the function \code{plot\_cumhaz} in order to easily
display cumulative hazard plots, given that the data follows a specified
parametric distribution. The functionality of this function is nearly
identical to that of \code{plot\_haz}, with the only distinction being
that it plots cumulative hazard curves instead of hazard curves.

\textbf{Insert brief interpretation}

\begin{Schunk}

\includegraphics{Simple_Survival_Analysis_files/figure-latex/unnamed-chunk-6-1} \end{Schunk}

\hypertarget{computing-survival-probabilities-and-summary-statistics}{%
\subsection{Computing Survival Probabilities and Summary
Statistics}\label{computing-survival-probabilities-and-summary-statistics}}

While viewing plots such as those explained above are very useful in
survival analysis, they only tell half of the story. In order to carry
out a complete survival analysis, we must also compute statistics in
order to supplement our plots. Some of the most common statistics
utilized in parametric survival analysis are survival probabilities and
typical summary statistics such as the mean, median, standard deviation,
and percentiles of survival time.

\hypertarget{computing-survival-probabilites}{%
\subsection{Computing Survival
Probabilites}\label{computing-survival-probabilites}}

Being able to compute survival probabilites is especially of interest
because it estimates the probability that a subject will not have
experienced the event of interest beyond a specified time \(t\). We
developed the function \code{surv\_prob} to compute probability of
survival beyond time \(t\), given that the data follows a specified
parametric distribution. The first argument of this function is a data
frame to be refrenced, containing a time column and censor column. The
second argument is a string name for a distribution that has a
corresponding density function and a distribution function. Some
examples of the distributions that can be used here are \code{norm},
\code{lnorm}, \code{exp}, \code{weibull}, \code{logis}, \code{llogis},
and \code{gompertz}. The third argument is the time at which suvival is
to be computed, which is input as a scalar quantity. The 4th argument is
the string name of the time column of the dataframe. This argument
defaults to \code{Time}. Similarly, the final argument is the string
name of the censor column of the dataframe, which defaults to
\code{Censor}. The censor column must be a numeric indicator variable
where complete times correspond to a value of 1, and censored times
correspond to a value of 0. An example of this function is shown below.

\begin{Schunk}
\begin{Sinput}
library(survival)
library(parmsurvfit)
data("rats")
surv_prob(rats, "lnorm", 110, time="time", censor="status")
\end{Sinput}
\begin{Soutput}
#> P(T > 110) = 0.7948027
\end{Soutput}
\end{Schunk}

As seen in the output from the function above, utilizing the rats
dataset available in the \CRANpkg{survival} package and fiting a
log-normal parametric distribution to the data, the estimated
probability that a rat survives beyond 110 days is 0.7948, or roughly
80\%.

\hypertarget{computing-summary-statistics}{%
\subsubsection{Computing Summary
Statistics}\label{computing-summary-statistics}}

Another useful form of output that we believed would be useful to also
have in R is a table of summary statistics. Summmary statistics that are
typically included are the mean, standard deviation, median, and IQR.
The surv\_summary function that we developed estimates various summary
statistics, including mean, median, standard deviation, and percentiles
of survival time given that the data follows a specified parametric
distribution. This function also supports the option to provide seperate
summary statistics for each level of a grouping variable, if desired.

\begin{Schunk}
\begin{Sinput}
library(parmsurvfit)
library(survival)
data("rats")
surv_summary(rats, "lnorm", time="time", censor = "status", by="sex")
\end{Sinput}
\begin{Soutput}
#> 
#> 
#> For level = f 
#> meanlog      4.876678
#> sdlog        0.487993
#> Log Liklihood    -247.0791
#> AIC      498.1581
#> BIC      504.1794
#> Mean     147.7833
#> StDev        76.63145
#> First Quantile   94.39914
#> Median       131.1941
#> Third Quantile   182.3311
#> 
#> For level = m 
#> meanlog      6.201628
#> sdlog        0.7477529
#> Log Liklihood    -18.39156
#> AIC      40.78312
#> BIC      46.80439
#> Mean     652.7505
#> StDev        564.981
#> First Quantile   298.0544
#> Median       493.5518
#> Third Quantile   817.2781
\end{Soutput}
\end{Schunk}

As seen above, after specifying the grouping variable of sex, two
seperate tables were produced. We can see that the mean log survival
time for female rats was smaller than the mean log survival time for
male rats. The standard deviation of log survival time for female rats
was also much smaller than that of male rats.

\hypertarget{summary}{%
\subsection{Summary}\label{summary}}

This file is only a basic article template. For full details of
\emph{The R Journal} style and information on how to prepare your
article for submission, see the
\href{https://journal.r-project.org/share/author-guide.pdf}{Instructions
for Authors}.

\bibliography{RJreferences}


\address{%
Victor Wilson\\
California Polytechnic State University, San Luis Obispo - Statistics
Department\\
\\
}
\href{mailto:victorjw26@yahoo.com}{\nolinkurl{victorjw26@yahoo.com}}

\address{%
Ashley Jacobson\\
California Polytechnic State University, San Luis Obispo - Statistics
Department\\
\\
}
\href{mailto:ashleypjacobson@gmail.com}{\nolinkurl{ashleypjacobson@gmail.com}}

\address{%
Shannon Pileggi\\
California Polytechnic State University, San Luis Obispo - Statistics
Department\\
\\
}
\href{mailto:spileggi@calpoly.edu}{\nolinkurl{spileggi@calpoly.edu}}

